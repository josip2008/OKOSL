\documentclass[12pt,a4paper]{article}
\usepackage[croatian]{babel}
\usepackage[utf8]{inputenc}
\usepackage[top=20mm]{geometry}
\usepackage{enumerate}
\newcommand{\shell}[1]{\texttt{\textbf{#1}}}
\renewcommand*{\familydefault}{\sfdefault}
\renewcommand*{\sfdefault}{lmss}
\begin{document}
	\title{Domaća zadaća - 05\\{\small Osnove korištenja operacijskog sustava Linux}\vspace{-2em}}
	\maketitle
	Za \textbf{svaki} zadatak treba napisati bash skriptu s rješenjem zadatka. Jedna se točka zadatka \textbf{može riješiti s više redova i naredbi} u skripti. \\
	
	\subsection*{Zadatak 1}
	\begin{itemize}
	    \item Ispišite \textbf{sve} procese koje je pokrenuo trenutni korisnik (i \textbf{samo} njih). Kao i uvijek, učinite to dinamički\footnote{Ovisno o pristupu koji odaberete, ako vam je korisničko ime dulje od 8 znakova, ono bi moglo biti skraćeno na format "prvih\_7\_znakova+". U tom slučaju nađite način da doskočite tome.}, a ne hardkodiranjem svog korisničkog imena.
	    \item Sortirajte prethodni ispis po zauzeću memorije.
	    \item Ispišite stablo procesa za trenutnog korisnika.

	\end{itemize}

	\subsection*{Zadatak 2}
	Sljedeći se zadatak odnosi na skriptu \shell{even\_bash.sh}. To je skripta koja prima jedan cjelobrojni argument, a ispisuje je li on paran ili neparan.
	\begin{itemize}
		\item Koristeći ljusku \shell{bash} i tu skriptu, provjerite parnost brojeva 27 i 42. Skriptu pokrenite sekvencom \shell{./}, a ne naredbom \shell{bash}.
		\item Nakon toga, pokrenite neku ljusku koja nije \shell{bash} (npr. \shell{sh}, koja je dostupna na većini sustava - jednostavno pokretanjem naredbe \shell{sh}). Iz nove ljuske, pokušajte na isti način pokrenuti skriptu. U komentar napišite u čemu je problem.
	\end{itemize}

	\subsection*{Zadatak 3}
	Sljedeći se zadatak odnosi na C program signal\_handler.c.
	\begin{enumerate}
		\item Signali SIGTERM vs. SIGKILL
		\begin{itemize}
		    \item Prevedite taj program (npr. \shell{gcc signal\_handler.c}) i pokrenite ga.
		    \item Pronađite PID tog programa. Dinamičko rješenje nije nužno, ali bit će dodatno nagrađeno (jer se rezultat koristi u sljedećoj točki). U svakom slučaju, postupak opišite u komentarima.
		    \item Pokušajte "ubiti" taj proces naredbom \shell{kill <pid>}. Zašto niste uspjeli?
		    \item Pokušajte ga sada ubiti slanjem adekvatnog signala. Zašto ste sad uspjeli, iako je naoko situacija u kôdu ista?
		  \end{itemize}
		\item Exit codes
		\begin{itemize}
			\item Pokrenite program i pošaljite mu SIGUSR1.
		    \item Pokrenite program i pošaljite mu SIGINT. Ponovno pokrenite program i pošaljite mu SIGUSR2. 
		    \item Istražite što u ljusci radi sekvenca \shell{\$?} i potom u komentaru opišite kako možemo (osim čitanjem ispisane izlazne poruke) iz ljuske utvrditi je li izvođenje programa uspjelo ili ne. Misli se na semantiku izvođenja, u smislu "je li obavljeno što je trebalo biti obavljeno".
		\end{itemize}
	\end{enumerate}
\end{document}
