\documentclass[table,usenames,dvipsnames]{beamer}

\usepackage[english]{babel}
\usepackage[utf8]{inputenc}
\usepackage{listings}
\usepackage{datetime}
\usepackage{graphics}
\usepackage{fancybox}
\usepackage{color}
\usepackage[normalem]{ulem}
\usepackage{tikz}
\usepackage{verbatim}
\usepackage{varwidth}
\usetikzlibrary{shapes,arrows}
\usetheme{CambridgeUS}
\usecolortheme{seagull}
% Changing of bullet foreground color not possible if {itemize item}[ball]
\DefineNamedColor{named}{Purple}{cmyk}{0.52,0.97,0,0.55}
\setbeamertemplate{itemize item}[triangle]
\setbeamercolor{title}{fg=Purple}
\setbeamercolor{frametitle}{fg=Purple}
\setbeamercolor{itemize item}{fg=Purple}
\setbeamercolor{section number projected}{bg=Purple,fg=white}
\setbeamercolor{subsection number projected}{bg=Purple}

\renewcommand{\dateseparator}{.}
\newcommand{\todayiso}{\twodigit\day \dateseparator \twodigit\month \dateseparator \the\year}
\newcommand{\shell}[1]{\texttt{#1}}
\newcommand{\MyCBox}[1]{%
  \fcolorbox{black}{gray!30}{\begin{varwidth}{\dimexpr\linewidth-2\fboxsep}#1
\end{varwidth}}}
\definecolor{LightGray}{gray}{0.9}

\title{Osnove korištenja operacijskog sustava Linux}
\subtitle{08. Zamjenski znakovi i regularni izrazi} 
\author[Dino Lukman i Goran Cetušić]{Dino Lukman i Goran Cetušić\\{\small Nositelj: dr. sc. Stjepan Groš}}
\institute[FER]{Sveučilište u Zagrebu \\
				Fakultet elektrotehnike i računarstva}
				
\date{\todayiso}

\begin{document}
    %\beamerdefaultoverlayspecification{<+->}
{
\setbeamertemplate{headline}[] % still there but empty
\setbeamertemplate{footline}{}

\begin{frame}
\maketitle
\end{frame}
}

\begin{frame}
\frametitle{Sadržaj}
\tableofcontents
\end{frame}


\section{Regularni izrazi}
\begin{frame}[t]
\frametitle{Regularni izrazi}
\begin{itemize}
  \item \textbf{Primjer}
  \item[] Izlistati sve datoteke u \shell{/usr/bin} direktoriju koje započinju s ab, bi ili ci
  \begin{itemize}
	  \item Vrlo teško sa zamjenskim znakovima
  \end{itemize}
\end{itemize}
\vfill
\textbf{Regularni izrazi}
  \begin{itemize}
    \item Korištenje znakova i operatora te pravila regularnih izraza za 
          obradu teksta, pretragu, leksičku analizu, \ldots
    \item Moćno i kompleksno proširenje zamjenskih znakova
  \end{itemize}
\vfill
\end{frame}

\begin{frame}[t]
\frametitle{Regularni izrazi}
\framesubtitle{Sintaksa}
\textbf{Definiranje traženog niza}
\begin{table}[h]
\rowcolors{1}{White}{LightGray}
\begin{tabular}{c l}
  \shell{x} & Znak \shell{x}\\
  \textasciicircum{} & Početak reda \\
  \shell{\$} & Kraj reda \\
  \shell{.} & Bilo koji znak \\
  \shell{[xy]} & Bilo koji znak u setu \\
  \shell{[\textasciicircum{}xy]} &  Invertirani set - Bilo koji znak koji nije u setu \\
  \textbackslash{}\shell{[} & Znak \shell{[} se interpretira kao traženi simbol, \\
  & a ne kao dio sintakse regularnog izraza \\
  \shell{|} & ILI operator \\
\end{tabular}
\end{table}
\end{frame}

\begin{frame}[t]
\frametitle{Regularni izrazi}
\framesubtitle{Sintaksa}
\textbf{Modifikator broja pojavljivanja niza}
\begin{table}[h]
\rowcolors{1}{White}{LightGray}
\begin{tabular}{c l}
  \shell{x*} & Izraz x se pojavljuje bilo koji broj puta ili nijednom\\
  \shell{x+} & Izraz x se pojavljuje najmanje jednom \\
  \shell{x?} & Izraz x se pojavljuje točno jednom ili nijednom \\
  \shell{x\{n\}} & Izraz x se pojavljuje točno n puta \\
  \shell{x\{n,m\}} & Izraz x se pojavljuje $n \le i \le m$ puta
\end{tabular}
\end{table}
\end{frame}
  
\begin{frame}[t]
\frametitle{Regularni izrazi}
\begin{itemize}
  \item \textbf{Primjer 1}
  \item[] Regularan izraz koji će zamjenjivati niz znakova \shell{foo} koji se nalazi na početku nekog retka datoteke
  \begin{itemize}
    \item[] \shell{\textasciicircum{}foo}
  \end{itemize}
  \item[]
  \item[] Regularan izraz koji će zamjenjivati niz znakova \emph{sigh}
        koji se nalazi na kraju nekog retka datoteke
  \begin{itemize}
    \item[] \shell{sigh\$}
  \end{itemize}
\end{itemize}
\end{frame}

\begin{frame}[t]
\frametitle{Regularni izrazi}
\begin{itemize}
  \item \textbf{Primjer 2}
  \item[] Linija koja ne završava sa slovom A 
  \begin{itemize}
    \item[] \shell{.*[\textasciicircum{}aA]\$}
  \end{itemize}
  \item[] Cijeli broj
  \begin{itemize}
    \item[] \shell{[0-9]+}
  \end{itemize}
  \item[] Varijabla u C-u
  \begin{itemize}
    \item[] \shell{[a-zA-Z]([a-zA-Z0-9])*}
  \end{itemize}
\end{itemize}
\vfill
\begin{itemize}
  \item \textbf{Primjer 3}
  \item[] Linija koja završava s tekstovima \shell{conf} ili \shell{log}
  \begin{itemize}
  	\item[] \shell{.*[conf|log]\$}
  \end{itemize}
  \item[] Linija u kojoj postoji troznamenkasti ili četveroznamenkasti broj
  \begin{itemize}
  	\item[] \shell{.*[0-9]\{3,4\}.*}
  \end{itemize}
\end{itemize}
\vfill
\end{frame}

\begin{frame}[t]
\frametitle{Regularni izrazi}
\begin{itemize}
	\item Prethodnim tablicama opisan je \textbf{prošireni regex}
	\item \textbf{Osnovni regex} ne sadrži modifikatore broja pojavljivanja niza niti operator ILI
\end{itemize}
\begin{itemize}
  \item Regularne izraze prihvaća mnoštvo programa
  \item Mi ćemo se pozabaviti naredbom \shell{grep}
  \begin{tabular}{l l}
    grep  &  Koristi osnovni regex\\
    egrep &  Koristi prošireni regex
  \end{tabular}
\end{itemize}
\end{frame}
  
\begin{frame}[t]
\frametitle{Regularni izrazi}
\begin{columns}[c]
\column{0.7\textwidth}
\begin{itemize}
  \item Pretpostavite kako je niz znakova s desne strane upisan u datoteku 
        \shell{regular}
  \item Odredite koji izlaz će dati sljedeći niz naredbi
  \begin{itemize}
    \item[] \shell{egrep a.e regular}
    \item[] \shell{egrep a.+e regular}
    \item[] \shell{egrep ab*e regular}
    \item[] \shell{egrep "(ab|cd)e" regular}
    \item[] \shell{egrep "(a|c).+e\$" regular}
  \end{itemize}
\end{itemize} 
\column{0.3\textwidth}
  \centering
  \MyCBox{abe \\ abbe \\ cde \\ 45a678 \\ ae \\ cababb \\ 12345} 
\end{columns}
\end{frame}

\begin{frame}[t]
\frametitle{Regularni izrazi}
\framesubtitle{Sintaksa}
\textbf{Klase znakova}
\begin{itemize}
  \item \shell{[:alnum:]} znakovi abecede ili brojevi. A-Za-z0-9
  \item \shell{[:alpha:]} znakovi abecede. A-Za-z 
  \item \shell{[:blank:]} praznina ili tabularni znak
  \item \shell{[:cntrl:]} kontrolni znakovi
  \item \shell{[:digit:]} brojevi. 0-9
  \item \shell{[:lower:]} mala slova abecede. a-z
  \item \shell{[:upper:]} velika slova abecede. A-Z
  \item \shell{[:space:]} praznina
  \item \shell{[:print:]} znakovi koji mogu biti ispisani. Podrazumijeva 
        znakove u ASCII rasponu 32 – 126 uključujući prazan znak.
\end{itemize}
\end{frame}

\begin{frame}[t]
\frametitle{Regularni izrazi}
\begin{itemize}
\item \textbf{Primjer 4}
\item[] Izlistajte linije iz datoteke \shell{file.txt} koje sadrže samo brojeve
\begin{itemize}
	\item Prvi način
	\item[] \shell{\$ grep [0-9]+ file.txt}
	\item Drugi način
	\item[] \shell{\$ grep [[:digit:]]+ file.txt}
\end{itemize}
\end{itemize}
\end{frame}

\begin{frame}[fragile]
\frametitle{Regularni izrazi}
\framesubtitle{Sintaksa}
\textbf{Označeni podizraz}
\begin{itemize}
	\item Omogućuje korištenje nađenog izraza u daljnjem radu neke naredbe
	\item Izraz se označava zatvaranjem kriterija u () zagrade
	\item Nađeni izraz se koristi pozivanjem \shell{$\backslash$1} u naredbi, gdje je \shell{1} redni broj označenog podizraza
\end{itemize}
\vfill
\vfill
\end{frame}


\begin{frame}
\frametitle{sed}
\begin{itemize}
	\item \textbf{S}tream \textbf{ED}itor
	\item Služi za filtriranje i transformaciju teksta
	\item Naredba vrlo širokih mogućnosti
	\item Suštinski jednostavan, ali dokumentacija nije baš \textit{user-friendly}
	\item Nekoliko naredbi, ali daleko najčešća \textbf{s} - \textit{substitute}

	\begin{itemize}
		\item \textbf{Primjer 5}
		\item[] Naredba \shell{sed} za filtriranje i obradu teksta
		\small \$ sed -r "s/windows/linux/" file.txt
	\end{itemize}
\end{itemize}
\end{frame}

\begin{frame}
\frametitle{sed - primjeri}
	\begin{itemize}
	    \item Nekoliko primjera:
		\begin{itemize}
			\item \textbf{Primjer 6}
			\small \$ sed -r "s/windows/linux/" file.txt
		\end{itemize}

		\item mijenja \textbf{samo prvo javljanje} riječi "windows" u "linux" u jednom retku
		\item navodnici nisu neophodni, ali su poželjni i dobra praksa
		\item prvi znak iza slova "s" je delimiter - može biti bilo što, ali najčešće /
	\end{itemize}
\end{frame}

\begin{frame}
    \begin{itemize}
		\begin{itemize}
			\item \textbf{Primjer 7}
			\small \$ sed -r "s/\textasciicircum{}Tekst ([[:digit:]])\. reda\$/Promijenili 
			smo \1. red/" file.txt
		\end{itemize}

		\item korištenjem \\1 (\\2, ...) može se sačuvati i ponovno iskoristiti određeni dio reda koji odgovara patternu

		\begin{itemize}
			\item \textbf{Primjer 8 - zamjena redoslijeda dvije riječi}
			\small \$ sed -r 's/([a-z]+) ([a-z]+)/\2 \1/' file.txt
		\end{itemize}

		\item ako se nakon zadnjeg delimitera doda "g", on će zamijeniti sva pojavljivanja te riječi u istom retku
	\end{itemize}
\end{frame}

\begin{frame}[t]
\frametitle{Literatura}
\begin{itemize}
	\item[] \shell{man 7 regex}
	\item[] \shell{man grep}
	\item[]
  \item[] \small\url{http://www.regexr.com}
  \item[] \small\url{http://regexone.com}
  \item[] \small\url{http://regex.learncodethehardway.org/book/}
  \item[] \small\url{http://www.rexegg.com}
  \item[] \small\url{http://www.regular-expressions.info/conditional.html} 
  \item[] \small\url{http://tldp.org/LDP/Bash-Beginners-Guide/html/chap_04.html}
\end{itemize}
\end{frame}

\end{document}
