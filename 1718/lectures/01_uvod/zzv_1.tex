\documentclass{beamer}

\usepackage[english]{babel}
\usepackage[utf8]{inputenc}
\usepackage{listings}
\usepackage{datetime}
\usepackage{graphics}
\usepackage{fancybox}
\usepackage{color}
\usepackage[normalem]{ulem}
\usepackage{hyperref}
\usetheme{CambridgeUS}
\usecolortheme{seagull}
% Changing of bullet foreground color not possible if {itemize item}[ball]
\DefineNamedColor{named}{Purple}{cmyk}{0.52,0.97,0,0.55}
\setbeamertemplate{itemize item}[triangle]
\setbeamercolor{title}{fg=Purple}
\setbeamercolor{frametitle}{fg=Purple}
\setbeamercolor{itemize item}{fg=Purple}
\setbeamercolor{section number projected}{bg=Purple,fg=white}
\setbeamercolor{subsection number projected}{bg=Purple}

\renewcommand{\dateseparator}{.}
\newcommand{\todayiso}{\twodigit\day \dateseparator \twodigit\month \dateseparator \the\year}
\newcommand{\shell}[1]{\texttt{#1}}
\title{Osnove korištenja operacijskog sustava Linux}
\subtitle{01. Zadaci za vježbu}
\author[Antun Aleksa, Josip Žuljević]{Antun Aleksa, Josip Žuljević\\{\small Nositelj: dr. sc. Stjepan Groš}}
\institute[FER]{Sveučilište u Zagrebu \\
				Fakultet elektrotehnike i računarstva}

\date{\todayiso}

\begin{document}
    %\beamerdefaultoverlayspecification{<+->}
{
\setbeamertemplate{headline}[] % still there but empty
\setbeamertemplate{footline}{}

\begin{frame}
\maketitle
\end{frame}
}

\begin{frame}[t]
\frametitle{Zadatak 1}
\begin{itemize}
  \item Prvi zadatak:
  \begin{itemize}
    \item Pozicioniraj se na Desktop
    \item Pozicioniraj se u \shell{/tmp}
    \item Napravite direktorij \shell{music} \emph{na radnoj površini} iz trenutne pozicije (bez korištenja \shell{cd}) i u njemu novi direktorij \shell{moja\_zika}
	\item Pozicioniraj se u \shell{music}
	\item Ispišite trenutni sadržaj tog direktorija
  \end{itemize}
\end{itemize}
\end{frame}

\begin{frame}[fragile]
\frametitle{Zadatak 1 - rješenje}
    \begin{verbatim}
#!/bin/bash

cd ~/Desktop
echo "sad se nalazim u $PWD"
cd /tmp/
echo "premjestio sam se u $PWD"
mkdir -p ~/Desktop/music/moja_zika
ls ~/Desktop/*
cd ~/Desktop/music/
ls
    \end{verbatim}
\end{frame}

\begin{frame}[t]
\frametitle{Zadatak 2}
\begin{itemize}
  \item Drugi zadatak:
  \begin{itemize}
    \item Pozicioniraj se na desktop
    \item Napravi direktorije "a", "b", "c" (jednom naredbom)
    \item U direktoriju "a" napravi direktorij "aa"
	\item Pozicioniraj se u direktorij "aa"
	\item Ispišite trenutni položaj
	\item Vratite se u direktorij "a" (korištenjem izraza za roditeljski direktorij)
	\item Ispišite trenutni položaj
  \end{itemize}
\end{itemize}
\end{frame}

\begin{frame}[fragile]
\frametitle{Zadatak 2 - rješenje}
    \begin{verbatim}
#!/bin/bash

cd ~/Desktop
echo "sad se nalazim u $PWD"
mkdir {a,b,c,d}
echo "trenutno mi $PWD izgleda ovako:"
ls
mkdir a/aa
cd aa
echo $PWD
cd ../.
echo $PWD
    \end{verbatim}
\end{frame}

\begin{frame}[t]
\frametitle{Zadatak 3}
\begin{itemize}
	\item Treći zadatak
	\begin{itemize}
		\item Pozicionirajte se u matični direktorij
		\item Napravite \emph{skriveni} direktorij \shell{OKOSL}
		\item Napravite još dva direktorija u njemu
		\item Pozicionirajte se u \shell{/tmp}
		\item Ne mijenjajući trenutni direktorij ispišite sadržaj novih foldera
	\end{itemize}
\end{itemize}
\end{frame}

\begin{frame}[fragile]
\frametitle{Zadatak 3 - rješenje}
    \begin{verbatim}
#!/bin/bash

cd ~
mkdir .OKOSL
mkdir .OKOSL/{prvi,drugi}
ls .OKOSL/*
cd /tmp
echo "trenutno se nalazim u $PWD i ispisujem ls ~/.OKOSL"
ls ~/.OKOSL
    \end{verbatim}
\end{frame}

\begin{frame}[t]
\frametitle{Zadatak 4}
\begin{itemize}
	\item Četvrti zadatak
	\begin{itemize}
		\item Otvorite \shell{man} stranice za \shell{ls}
		\item Pronađite opciju koja ispisuje sadržaj svih poddirektorija
		\item Ispišite sadržaje svih direktorija unutar \shell{/home}
		\item Ispišite sadržaje svih direktorija \emph{uključujući i skrivene} unutar \shell{/home}
		\item Odredite veličinu datoteke \shell{/sbin/wpa\_supplicant} \emph{u megabajtima}
	\end{itemize}
\end{itemize}
\end{frame}

\begin{frame}[fragile]
\frametitle{Zadatak 4 - rješenje}
    \begin{verbatim}
#!/bin/bash

man ls
ls ~
ls -al ~
ls -lh /sbin/wpa_supplicant
    \end{verbatim}
\end{frame}

\end{document}
