% sredit sadrzaj - ovo ne lici ni na sto
% pogreske - slajd preusmjeravanje u datoteku - ide u /tmp/test
% nezgodno ispisuje \ u \shell{}, mozda verbatim

\documentclass{beamer}

\usepackage[english]{babel}
\usepackage[utf8]{inputenc}
\usepackage{listings}
\usepackage{datetime}
\usepackage{graphics}
\usepackage{fancybox}
\usepackage{color}
\usepackage[normalem]{ulem}
\usepackage{tikz}
\usepackage{verbatim}
\usetikzlibrary{shapes,arrows,positioning}
\usetheme{CambridgeUS}
\usecolortheme{seagull}
% Changing of bullet foreground color not possible if {itemize item}[ball]
\DefineNamedColor{named}{Purple}{cmyk}{0.52,0.97,0,0.55}
\setbeamertemplate{itemize item}[triangle]
\setbeamercolor{title}{fg=Purple}
\setbeamercolor{frametitle}{fg=Purple}
\setbeamercolor{itemize item}{fg=Purple}
\setbeamercolor{section number projected}{bg=Purple,fg=white}
\setbeamercolor{subsection number projected}{bg=Purple}

\renewcommand{\dateseparator}{.}
\newcommand{\todayiso}{\twodigit\day \dateseparator \twodigit\month \dateseparator \the\year}
\newcommand{\shell}[1]{\texttt{#1}}

\title{Osnove korištenja operacijskog sustava Linux}
\subtitle{03. Pretraživanja, filtri i cjevovodi}
\author[Nino Nikola Stanušić i Goran Cetušić]{Nino Nikola Stanušić i Goran Cetušić\\{\small Nositelj: dr. sc. Stjepan Groš}}
\institute[FER]{Sveučilište u Zagrebu \\
				Fakultet elektrotehnike i računarstva}

\date{\todayiso}

\begin{document}
    %\beamerdefaultoverlayspecification{<+->}
{
\setbeamertemplate{headline}[] % still there but empty
\setbeamertemplate{footline}{}

\begin{frame}
\maketitle
\end{frame}
}

\begin{frame}
\frametitle{Sadržaj}
\tableofcontents
\end{frame}

\section{Pojam standardnog izlaza i ulaza}
\begin{frame}[t]
\frametitle{Pojam standardnog izlaza i ulaza}
\begin{itemize}
  \item Svaki program na Linuxu ima definirane sljedeće ulaze i izlaze
  \begin{itemize}
    \item Standardni ulaz (stdin)
    \item Standardni izlaz (stdout)
    \item Standardni izlaz za greške (stderr)
  \end{itemize}
  \item Svi ti ulazi i izlazi su vezani na terminal
  \begin{itemize}
    \item Ako ih nismo preusmjerili uz pomoć specijalnih operatora
  \end{itemize}
  \item Svi su predstavljeni kao datoteke
  \begin{itemize}
    \item Imaju svoje opisnike datoteka (engl. \emph{file descriptor})
    \item Označeni su brojevima 0, 1 i 2
  \end{itemize}
\end{itemize}
\end{frame}



\begin{frame}[t]
\frametitle{Primjer naredbe \shell{cat}}
\begin{itemize}
  \item Grafička ilustracija izlaza i ulaza
  \vspace{1cm}
  \tikzstyle{plaintext}=[thick]
  \tikzstyle{block}=[rectangle,draw=black!50,fill=black!20,thick]
  \begin{tikzpicture}[auto, >=latex',node distance=4.5cm]
    \node[plaintext]  (a)               {tipkovnica};
    \node[block]      (b) [right of=a,inner sep=1cm,inner ysep=4mm]  {cat};
    \node[plaintext]  (c) [right of=b]  {zaslon};
    \draw[->] (a) edge node {stdin} (b);
    \draw[->] (b) edge node {stdout} (c);
  \end{tikzpicture}
  \vspace{1cm}
  \item Naredba \shell{cat} je filter!
  \begin{itemize}
    \item Preuzima nešto na ulazu
    \item Filtrira preuzete podatke
    \item Proslijeđuje rezultat na standardni izlaz
  \end{itemize}
\end{itemize}
\end{frame}

\section{Filtri}
\begin{frame}[t]
\frametitle{Filtri}
\begin{itemize}
  \item Približna definicija filtara bi mogla biti
  \begin{itemize}
    \footnotesize
    \item[] Svaki program koji ulazne podatke čita sa standardnog ulaza,
            obrađuje ih na odgovarajući način te rezultat obrade
            proslijeđuje na standarni izlaz
    \normalsize
  \end{itemize}
  \item Moguće je vrlo složeno kombiniranje filtara uz pomoć odgovarajućih
        operatora
  \begin{itemize}
    \item Ostaje pitanje, zbog čega standardni izlaz za greške?
    \item Služi kako bi smo mogli uočiti greške u podacima
  \end{itemize}
\end{itemize}
\end{frame}

\section{Preusmjeravanje}
\begin{frame}[t]
\frametitle{Preusmjeravanje u datoteku (1)}
\begin{itemize}
  \item Kako bi preusmjerili izlaz neke naredbe u datoteku koristimo
        operator \shell{">"}
  \begin{itemize}
    \item Pogledajmo razliku između izvršavanja sljedeće dvije naredbe:
    \begin{itemize}
      \item[] \shell{\$ ls -l /}
      \item[] \shell{\$ ls -l / > /tmp/test}
    \end{itemize}
  \end{itemize}
  \item Zaključak
  \begin{itemize}
    \item Druga varijanta naredbe preusmjerava izlaz u datoteku
          \shell{/tmp/test}
    \item Što ako datoteka već postoji?
  \end{itemize}
\end{itemize}
\end{frame}

\begin{frame}[t]
\frametitle{Preusmjeravanje u datoteku (2)}
\begin{itemize}
  \item Što ako želimo dodati sadržaj u datoteku?
  \begin{itemize}
    \item Koristit ćemo operator \shell{>>}
  \end{itemize}
  \item Primjer
  \begin{itemize}
    \item Potrebno je u jednu datoteku dobiti ispis direktorija
          \shell{/bin, /sbin, /usr/bin} i \shell{/usr/sbin}
    \begin{itemize}
      \item Napomena: za svaki direktorij mora se posebno pozivati
              naredba \shell{ls}!
    \end{itemize}
  \end{itemize}
\end{itemize}
\end{frame}

\begin{frame}[t]
\frametitle{Preusmjeravanje u datoteku (3)}
\begin{itemize}
	\item Riješenje
	\begin{itemize}

	\item[] \shell{\$ ls -l /tmp/}
	\item[] \shell{\$ ls -l /bin/ >> /tmp/output}
	\item[] \shell{\$ ls -l /sbin/ >> /tmp/output}
	\item[] \shell{\$ ls -l /usr/bin/ >> /tmp/output}
	\item[] \shell{\$ ls -l /usr/sbin/ >> /tmp/output}

	\end{itemize}
\end{itemize}
\end{frame}


\begin{frame}[t]
\frametitle{Preusmjeravanje iz datoteke}
\begin{itemize}
  \item Ako želimo sadržaj neke datoteke preusmjeriti u neki program,
        koristimo operator \shell{<}
  \item Recimo, želimo li pogledati sadržaj datoteke \shell{/etc/passwd},
        to možemo obaviti na sljedeći način:
  \begin{itemize}
    \item[] \shell{\$ cat < /etc/passwd}
    \item[] {\small Kod ove naredbe efekt je isti kao i bez korištenja \shell{<}}
  \end{itemize}
	\item \shell{<} je korisniji kod naredbi koje rade isključivo sa \shell{stdin} ulazom
\end{itemize}
\end{frame}

\begin{frame}[t]
\frametitle{Preusmjeravanje ulaza i izlaza (1)}
\begin{itemize}
  \item Moguće je istovremeno preusmjeravanje izlaza i ulaza
  \item Primjer kopiranja datoteke \shell{/etc/passwd} u
        \shell{/tmp/passwd} uz pomoć naredbe \shell{cat}
  \begin{itemize}
    \item[] \shell{\$ cat < /etc/passwd > /tmp/passwd}
    \item[] \shell{\$ cat > /tmp/passwd < /etc/passwd}
  \end{itemize}
  \item Nije bitan redoslijed operatora preusmjeravanja
\end{itemize}
\end{frame}

\begin{frame}[t]
	\frametitle{Preusmjeravanje ulaza i izlaza (2)}
	\begin{itemize}
		\item Vrlo često programi primaju putanju do ulazne ili izlazne datoteke kroz opcije
		\item Oznaka \shell{-} u tom kontekstu predstavlja \shell{stdin} ili \shell{stdout}
	\end{itemize}
	\begin{itemize}
		\item Primjer
		\item[] \shell{wget -O - http://ftp.hr.debian.org/README}
	\end{itemize}
\end{frame}

\begin{frame}[t]
\frametitle{Preusmjeravanje između programa (1)}
\begin{itemize}
  \item Do sada smo isključivo obavljali preusmjeravanje iz i u datoteku
  \begin{itemize}
    \item Ograničena funkcionalnost
    \item Loša efikasnost, ako podatke između programa prebacujemo preko
          datoteka
  \end{itemize}
  \item Način povezivanja programa je upotrebom operatora \shell{|} (engl.
        \emph{pipe})
  \item[]
  \item Primjer
  \begin{itemize}
  	\item Ispisati prvih pet linija ispisa direktorija \shell{/lib}
    \item[] \shell{\$ ls -l /lib/ | head -5}
  \end{itemize}
  \item Primjer
  \begin{itemize}
  	\item Ispisati sadržaj direktorija /usr/share u \textit{pageru}
  	\item[] \shell{\$ ls -l /usr/share | less}
  \end{itemize}
\end{itemize}
\end{frame}

\begin{frame}[t]
\frametitle{Preusmjeravanje između programa (2)}
\begin{itemize}
  \item Efektivno, postigli smo sljedeće
  \vspace{1cm}
  \tikzstyle{plaintext}=[thick]
  \tikzstyle{block}=[rectangle,draw=black!50,fill=black!20,thick,
                     inner sep=0.5cm, inner ysep=4mm]
  \begin{tikzpicture}[auto, >=latex',node distance=3cm]
    \node[block]      (a)               {ls -l /lib};
    \coordinate[right=2cm of a]  (b)  {};
    \node[block]  (c) [right of=b]  {head -5};
    \node[plaintext]  (d) [right of=c]  {zaslon};
    \draw[->] (a) -- node {stdout} (b);
    \draw[->] (b) -- node {stdin} (c);
    \draw[->] (c) -- node {} (d);
  \end{tikzpicture}
  \vspace{1cm}
  \item Zadatak
  \begin{itemize}
    \item Odredite koji je 10. zapis u datoteci \shell{/etc/group}
  \end{itemize}
\end{itemize}
\end{frame}

\begin{frame}[t]
\frametitle{Standardni izlaz za greške (1)}
\begin{itemize}
  \item Što je sa standardnim izlazom za greške?
  \item Pogledajmo ovu naredbu:
  \begin{itemize}
    \item[] \shell{\$ ls -l /lib1/}
    \item[] \shell{\$ ls -l /lib1/ > /tmp/lib5}
  \end{itemize}
  \item Što bi se trebalo dogoditi?
  \item Što se dogodilo?
  \begin{itemize}
    \item Zašto?
  \end{itemize}
\end{itemize}
\end{frame}

\begin{frame}[t]
\frametitle{Standardni izlaz za greške (2)}
\begin{itemize}
  \item Izlaz za greške možemo preusmjeriti operatorom \shell{2>}
  \begin{itemize}
    \item[] \shell{\$ ls -l /lib1/ 2> /tmp/lib5}
  \end{itemize}
  \item Taj operator ne preusmjerava standardni izlaz
  \begin{itemize}
    \item[] \shell{\$ ls -l /lib1/ 2> /tmp/lib5}
  \end{itemize}
  \vspace{1cm}
  \tikzstyle{plaintext}=[thick]
  \tikzstyle{block}=[rectangle,draw=black!50,fill=black!20,thick,
                     inner sep=0.5cm, inner ysep=4mm]
  \begin{tikzpicture}[auto, >=latex']
    \node[block]      (a)               {ls -l /lib1/};
    \node[plaintext]  (b) [right= 2cm of a]  {zaslon};
    \node[plaintext]  (c) [below=1cm of b]  {/tmp/lib5};
    \draw[->] (a) edge node {stdout} (b);
    \draw[->] (a)  |- node[near end] {stderr} (c);
  \end{tikzpicture}
\end{itemize}
\end{frame}

\begin{frame}[t]
\frametitle{Standardni izlaz za greške (3)}
\begin{itemize}%
  \item Pogledajmo ovu naredbu:
  \begin{itemize}
    \item[] \shell{\$ ls -l /lib1/ | head -5 2> /tmp/lib5}
  \end{itemize}
  \item Što bi se trebalo dogoditi?
  \item Što se dogodilo?
  \begin{itemize}
    \item Zašto?
    \item Kako biste osigurali spremanje svih ispisanih grešaka u datoteku?
  \end{itemize}
  \vspace{1cm}
  \tikzstyle{plaintext}=[thick]
  \tikzstyle{block}=[rectangle,draw=black!50,fill=black!20,thick,
  inner sep=0.5cm, inner ysep=4mm]
  \begin{tikzpicture}[auto, >=latex',node distance=3cm]
  \node[block]                 (a)               {ls -l /lib};
  \coordinate[right=2cm of a]  (b) {};
  \node[block]  (c) [right of=b]  {head -5};
  \node[plaintext]  (d) [right of=c]  {zaslon};
  \coordinate [below=1cm of c] (fit below) {};
  \draw[->] (a) -- node {stdout} (b);
  \draw[->] (b) -- node {stdin} (c);
  \draw[->] (c) -- node {} (d);
  \draw[->] (a) |- (fit below) -| (d);
  \end{tikzpicture}
\end{itemize}
\end{frame}

\begin{frame}[t]
\frametitle{Standardni izlaz za greške (4)}
\begin{itemize}
  \item Pokušajmo sada sljedeće
  \begin{itemize}
    \item[] \shell{\$ ls -l /lib1/ | head -5 >2 /tmp/lib5}
  \end{itemize}
  \item Što bi se trebalo dogoditi?
  \item Što se dogodilo?
  \begin{itemize}
    \item Zašto?
  \end{itemize}
\end{itemize}
\end{frame}

\begin{frame}[t]
\frametitle{Naprednije preusmjeravanje (1)}
\begin{itemize}
  \item Zajedničko preusmjeravanje stdout i stderr
  \item[] \shell{\$ ls \&> /dev/null}
  \item[ili] \shell{\$ ls >\& /dev/null}
  \item Preusmjeravanje stderr na stdin
  \item[] \shell{\$ ls 2>\&1}
\end{itemize}
\begin{itemize}
  \item Ljuska nudi vrlo napredne mogućnosti preusmjeravanja
  \begin{itemize}
    \item Proučiti man stranicu :)
  \end{itemize}
\end{itemize}
\end{frame}

\begin{frame}[t]
\frametitle{Naprednije preusmjeravanje (2)}
\begin{itemize}
	\item Primjeri s \shell{wget} {\footnotesize
	\item[] \shell{wget -O - http://ftp.hr.debian.org/README > out.txt}
	\item[] \shell{wget -O - http://ftp.hr.debian.org/README 2> err.txt}
	\item[] \shell{wget -O - http://ftp.hr.debian.org/README 2> err.txt > out.txt}
	\item[] \shell{wget -O - http://ftp.hr.debian.org/README \&> both.txt} }
\end{itemize}
\end{frame}

\begin{frame}[t]
\frametitle{Naprednije preusmjeravanje (3)}
\shell{tee}
\begin{itemize}
	\item Naredba preusmjerava \shell{stdin} na \shell{stdout} i zadanu datoteku
	\item \shell{tee --append}
	\begin{itemize}
		\item Nadopisuje na trenutni sadržaj datoteke
	\end{itemize}
\end{itemize}
\begin{itemize}
	\item Primjer
	\begin{itemize}
		\item \shell{ls -l /usr/share | tee lista.txt}
		\item \shell{ls -l /usr/share | tee -}
		\item[] {\footnotesize Ovisno o distribuciji, naredba rezultira ili dvostrukim  ili jednostrukim ispisom.}
	\end{itemize}
\end{itemize}
\end{frame}

\begin{frame}[fragile]
\frametitle{Here document (1)}
\begin{itemize}
	\item Koncept koji se vrlo često koristi u skriptama
	\item Dio skriptnog koda koji se tretira kao zasebna datoteka
	\item Omogućuje navođenja sadržaja cijelih datoteka \textit{inline} u skripti
\end{itemize}
\begin{itemize}
\item[] \begin{verbatim}
cat << EOF
Nova datoteka
linija 2
...
linija n
EOF
\end{verbatim}
\end{itemize}
\end{frame}

\begin{frame}[fragile]
\frametitle{Here document (2)}
\begin{itemize}
	\item Preusmjeravanje u datoteku
	\item[] \begin{verbatim}
	cat > file.txt << EOF
	Here doc
	EOF
	\end{verbatim}
	\item[ili] \begin{verbatim}
	cat << EOF > file.txt
	Here doc
	EOF
	\end{verbatim}
\end{itemize}
\begin{itemize}
	\item Na sličan način funkcioniraju i \textit{here strings}
	\item[] \begin{verbatim}
	cat <<< 'Text string'
	\end{verbatim}
\end{itemize}
\end{frame}

\begin{frame}[t]
\frametitle{Uvjetno izvođenje naredbi}
\begin{itemize}
  \item Naredbe se mogu ulančavati tako da se njihovo izvršavanje uvjetuje
  \begin{itemize}
    \item[] \shell{\$ true \&\& echo 1}
  \end{itemize}
  \item drugi niz naredbi izvršava se samo ako je naredba (naredbe) ispred
        operatora znakova \&\& vratila izlazni status "0"
  \item Primjeri
  \begin{itemize}
    \item[] \shell{\$ false \&\& echo 1}
    \item[] \shell{\$ false || echo 1}
  \end{itemize}
\end{itemize}
\end{frame}

\section{Često korištene filter naredbe}
\subsection{Brojanje znakova, riječi i linija}
\begin{frame}[t]
\frametitle{Brojanje znakova, riječi i linija (1)}
\begin{itemize}
  \item Za prebrojavanje znakova, riječi i linija u tekstu koristimo
        naredbu \shell{wc} (engl. \emph{word count})
  \item Broji riječi, linije i znakove u datoteci ili na standardnom ulaz
  \item Primjer, broj znakova, riječi i linija u datoteci
        \shell{/etc/passwd}
  \begin{itemize}
    \item[] \shell{\$ wc /etc/passwd}
  \end{itemize}
  \item U ispisu prvo je broj linija, potom riječi i na kraju broj znakova
\end{itemize}
\end{frame}

\begin{frame}[t]
\frametitle{Brojanje znakova, riječi i linija (2)}
\begin{itemize}
  \item Opća sintaksa naredbe je
  \begin{itemize}
    \item[] \shell{wc [opcije] [<datoteka>]}
  \end{itemize}
  \item Opcije su
  \begin{itemize}
    \item[] \shell{-w} broji samo riječi
    \item[] \shell{-l} broji samo linije
    \item[] \shell{-c} broji samo znakove
  \end{itemize}
  \item Ako se ne navede datoteka, tada se brojanje obavlja u podacima koji
        pristižu na standardni ulaz, a izlaz je uvijek stdout
\end{itemize}
\end{frame}

\begin{frame}[t]
\frametitle{Brojanje znakova, riječi i linija (3)}
\begin{itemize}
  \item Zadatak 1
  \begin{itemize}
	\item Prebrojite linije u datoteci \shell{/usr/include/stdio.h} koristeći
	\begin{itemize}
		\item[a)] naredbu \shell{wc} i navođenjem datoteke izravno
		\item[b)] operator preusmjeravanja \shell{stdin}
	\end{itemize}
	\item Razmislite i prokomentirajte rezultat
  \end{itemize}
  \item[]
  \item Zadatak 2
  \begin{itemize}
    \item Koliko datoteka i direktorija ima u direktoriju \shell{/usr/bin},
          a koliko ih je u \shell{/usr/include}?
  \end{itemize}
\end{itemize}
\end{frame}

\subsection{Pretraživanje tekstualnog sadržaja}
\begin{frame}[t]
\frametitle{Pretraživanje teksta (1)}
\begin{itemize}
  \item Za pretraživanje tekstualnih datoteka koristi se naredba
        \shell{grep}
  \begin{itemize}
    \item Traži uzorak u datoteci ili u podacima sa standardnog ulaza
    \item Ispisuje na zaslon liniju u kojoj je uzorak pronađen
  \end{itemize}
  \item Mnoštvo opcija koje omogućavaju promjenu ponašanja
\end{itemize}
\end{frame}

\begin{frame}[t]
\frametitle{Pretraživanje teksta (2)}
\begin{itemize}
  \item Primjer upotrebe
  \begin{itemize}
    \item Tražimo niz “root” u datoteci \shell{/etc/passwd}
    \item[] \shell{\$ grep root /etc/passwd}
  \end{itemize}
  \item U općem slučaju, naredba ima sintaksu
  \begin{itemize}
    \item[] \shell{grep [opcije] traženi\_uzorak [<datoteka>]}
  \end{itemize}
  \item Neke često korištene opcije
  \begin{itemize}
    \item \shell{-v} ispisuje sve linije gdje se ne pojavljuje zadani
          uzorak
    \item \shell{-i} pretraživanje bez razlike u malim i velikim slovima
  \end{itemize}
\end{itemize}
\end{frame}

\begin{frame}[t]
\frametitle{Pretraživanje teksta (3)}
\begin{itemize}
  \item Zadaci
  \begin{itemize}
    \item Pronaći svoje korisničko ime u datoteci \shell{/etc/passwd}
    \item Potražiti konstantu SEEK\_SET u datoteci
          \shell{/usr/include/stdio.h}
    \item Pronaći sve datoteke u \shell{/usr/lib} koje sadrže X11 u svom
          nazivu
  \end{itemize}
\end{itemize}
\end{frame}

\subsection{Sortiranje tekstualnog sadržaja}
\begin{frame}[t]
\frametitle{Sortiranje tekstualnih linija (1)}
\begin{itemize}
  \item Za sortiranje tekstualnih i numeričkih podataka koristimo naredbu
        \shell{sort}
  \item Primjeri
  \begin{itemize}
    \item Sortirajmo datoteku \shell{/usr/include/stdio.h}
    \item[] \shell{\$ sort /usr/include/stdio.h}
    \item Sortirajmo sve linije iz \shell{/usr/include/stdio.h} koje sadrže
          riječ \#define
    \item[] \shell{\$ grep \textbackslash{}\#define /usr/include/stdio.h |
                   sort}
  \end{itemize}
\end{itemize}
\end{frame}

\begin{frame}[t]
\frametitle{Sortiranje tekstualnih linija (2)}
\begin{itemize}
  \item Sintaksa naredbe sort je
  \begin{itemize}
    \item[] \shell{sort [opcije] [<datoteka>]}
  \end{itemize}
  \item Ako se datoteka ne navede tada se sortiraju podaci koji pristižu sa
        standardnog ulaza. Izlazi se sa  \^{}D  nakon sto su uneseni svi elementi.
  \item Rezultat se ispisuje na standardni izlaz
\end{itemize}
\end{frame}

\subsection{Uklanjanje duplih linija}
\begin{frame}[t]
\frametitle{Uklanjanje duplih linija (1)}
\begin{itemize}
  \item Za traženje i manipulaciju duplim linijama koristi se naredba
        \shell{uniq}
  \begin{itemize}
    \item Sintaksa naredbe je
    \item[] \shell{uniq [opcije] [<datoteka>]}
  \end{itemize}
  \item Češće korištene opcije su
  \begin{itemize}
    \item[] \shell{-u} Ispisuje linije koje se ne ponavljaju
    \item[] \shell{-d} Ispisuje linije koje se ponavljaju barem jednom
    \item[] \shell{-c} Ispisuje za svaku liniju koliko puta se ponavlja
  \end{itemize}
\end{itemize}
\end{frame}

\begin{frame}[t]
\frametitle{Uklanjanje duplih linija (2)}
\begin{itemize}
  \item Linije moraju biti sortirane prije korištenja ove naredbe!
  \item Primjer, provjerimo ima li duplih linija u jednoj od datoteka u direktoriju
        \shell{/usr/share/dict/}
  \begin{itemize}
    \item[] npr. \shell{\$ uniq -d /usr/share/dict/words}
  \end{itemize}
\end{itemize}
\end{frame}

\subsection{Izdvajanje pojedinih polja linije}
\begin{frame}[t]
\frametitle{Izdvajanje pojedinih polja linije (1)}
\begin{itemize}
  \item Za izdvajanje pojedinih polja u liniji koristimo naredbu
        \shell{cut}
  \item Podrazumijevani znak za razdvajanje polja je razmak
  \begin{itemize}
    \item Primjer datoteke s poljima \shell{/etc/passwd}
    \item Primjer, ispis korisničkih imena na sustavu
    \item[] \shell{\$ cut -f1 -d: /etc/passwd}
  \end{itemize}
\end{itemize}
\end{frame}

\begin{frame}[t]
\frametitle{Izdvajanje pojedinih polja linije (2)}
\begin{itemize}
  \item Sintaksa naredbe je sljedeća
  \begin{itemize}
    \item[] \shell{cut [opcije] [<datoteka>]}
  \end{itemize}
  \item Neke češće korištene opcije su
  \begin{itemize}
    \item[] \shell{-d} definira znak koji razgraničava polja
    \item[] \shell{-f} definira polja koja je potrebno propustiti na izlaz
    \item[] \shell{-c} ispisuje po kolonama
  \end{itemize}
  \item Ako nije navedena datoteka tada se obrada obavlja na podacima sa
        standardnog ulaza
\end{itemize}
\end{frame}

\begin{frame}[t]
\frametitle{Izdvajanje pojedinih polja linije (3)}
\begin{itemize}
  \item Primjer izrezivanja po kolonama
  \begin{itemize}
    \item[] \shell{\$ cut -c2-5,10-12 /etc/passwd}
  \end{itemize}
  \item[]
  \item Zadatak
  \begin{itemize}
    \item Potrebno je generirati popis svih različitih vrijednosti koje se
          javljaju u zadnjem polju datoteke \shell{/etc/passwd}
    \item Koliko puta se svaka vrijednost ponavlja?
  \end{itemize}
\end{itemize}
\end{frame}

\section{Pretraživanje datotečnog sustava}
\begin{frame}[t]
\frametitle{Pretraživanje datotečnog sustava (1)}
\begin{itemize}
  \item Za pretraživanje datoteka i direktorija koristimo naredbu
        \shell{find}
  \begin{itemize}
    \item Vrlo kompleksna naredba s nizom mogućnosti
  \end{itemize}
  \item Sintaksa naredbe je
  \begin{itemize}
    \item[] \shell{find [opcije] [<direktoriji>] <uvjeti>}
  \end{itemize}
  \item \shell{<direktoriji>} je popis direktorija koje je potrebno
        pretražiti
  \item \shell{<uvjeti>} definiraju uvjete pretraživanja
\end{itemize}
\end{frame}

\begin{frame}[t]
\frametitle{Pretraživanje datotečnog sustava (2)}
\begin{itemize}
  \item Primjeri nekih češće korištenih uvjeta
  \begin{tabular}{l l}
    \shell{-name <ime>} & Traži datoteku sa zadanim imenom  \\
    \shell{-type <tip>} & Traži datoteku zadanog tipa       \\
    \shell{-size <tip>} & Traži datoteku zadane veličine    \\
    \shell{-print}      & Ispisuje ime zadane datoteke
  \end{tabular}
\end{itemize}
\end{frame}

\begin{frame}[t]
\frametitle{Pretraživanje datotečnog sustava (3)}
\begin{itemize}
  \item Primjeri
  \begin{itemize}
    \item pronađimo datoteku \shell{passwd} u direktoriju \shell{/etc}
    \item[] \shell{\$ find /etc -name passwd}
    \item pronađimo sve datoteke u \shell{/etc} veće od 1M
    \item[] \shell{\$ find /etc -size +1M}
    \item pronađimo sve datoteke u \shell{/etc} manje od 1k
    \item[] \shell{\$ find /etc -size -1k}
  \end{itemize}
\end{itemize}
\end{frame}

\begin{frame}[t]
\frametitle{Pretraživanje datotečnog sustava (4)}
\begin{itemize}
  \item Primjeri
  \begin{itemize}
    \item pronađimo sve direktorije u \shell{/etc} direktoriju
    \item[] \shell{\$ find /etc -type d}
    \item pronađimo sve datoteke u \shell{/etc} direktoriju
    \item[] \shell{\$ find /etc -type f}
    \item pronađimo sve datoteke direktorije veće od 10k
    \item[] \shell{\$ find /etc -type d -a -size +10k}
    \item Operator \shell{-a} se podrazumijeva
  \end{itemize}
\end{itemize}
\end{frame}

\begin{frame}[t]
\frametitle{Pretraživanje datotečnog sustava (5)}
\begin{itemize}
  \item Primjeri
  \begin{itemize}
    \item pronađimo sve direktorije i datoteke veće od 1M u \shell{/etc}
          direktoriju
    \item[] \shell{\$ find /etc -type d -o -size +1M}
    \item isto kao i prethodni primjer, ali izlistava pronađene
          datoteke/direktorije
    \item[] \shell{\$ find /etc -type d -o -size +1M -ls}
  \end{itemize}
  \item[]
  \item[] \shell{man find}
  \begin{itemize}
    \item Pogledati opcije naredbe \shell{find} za MAC vremena
  \end{itemize}
\end{itemize}
\end{frame}

\begin{frame}[t]
\frametitle{Naredba \shell{whereis}}
\begin{itemize}
  \item Pronalazi gdje se nalazi izvorni kod naredbe, binarni zapis ili
        stranica priručnika
  \item Sintaksa
  \begin{itemize}
    \item \shell{whereis [<opcije>] <datoteka>}
  \end{itemize}
  \item Neke od opcija su
  \begin{itemize}
    \item[] \shell{-b} traži binarne zapise
    \item[] \shell{-m} traži stranice priručnika
    \item[] \shell{-s} traži izvorni kod
  \end{itemize}
\end{itemize}
\end{frame}

\begin{frame}[t]
\frametitle{Naredba \shell{locate} (1)}
\begin{itemize}
  \item Traži datoteku koja u svojem imenu sadrži zadani uzorak --
        preporučeno je njeno korištenje umjesto naredbe \shell{find}
  \begin{itemize}
    \item Manje opterećuje disk i brža je
    \item Koristi internu bazu podataka za pretragu
  \end{itemize}
  \item Sintaksa naredbe
  \begin{itemize}
    \item[] \shell{\$ locate [<opcije>] <uzorak>}
  \end{itemize}
  \item Korisna opcija je \shell{-i} za ignoriranje razlika u slovima
\end{itemize}
\end{frame}

\begin{frame}[t]
\frametitle{Naredba \shell{locate} (2)}
\begin{itemize}
  \item Primjer
  \begin{itemize}
    \item Želimo pronaći sve datoteke \shell{passwd}
    \item[] \shell{\$ locate passwd}
  \end{itemize}
  \item[]
  \item Zadatak
  \begin{itemize}
      \item Kreirati datoteku \shell{ovo\_je\_primjer} u matičnom direktoriju
	  \item Pokušati je pronaći naredbom \shell{locate}
	  \item Što se dogodilo?
	  \item Kako ispraviti problem?
  \end{itemize}
\end{itemize}
\end{frame}


\section{Pregled naredbi}

\begin{frame}[t]
\frametitle{Pregled naredbi}
\begin{tabular}{| l | c |} \hline
  \shell{>} & preusmjeravanja u datoteku \\ \hline
  \shell{>>} &  preusmjeravanje i nadodovanje\\ \hline
  \shell{<} & preusmjeravanje iz datoteke \\ \hline
  \shell{|} & cijevovod \\ \hline
  \shell{wc} & brojanje znakova, rijeci i linija \\ \hline
  \shell{grep} & pretrazivanje teksta \\ \hline
  \shell{sort} & sortiranje \\ \hline
  \shell{uniq} & uklanjnanje duplih linija \\ \hline
  \shell{cut} & izdvajanje pojedinih linija \\ \hline
  \shell{find} & pretrazivanje datoteka i direktorija \\ \hline
  \shell{locate} & pretrazivanje datoteka i direktorija \\ \hline
  \shell{whereis} & pronalazi izvorni kod naredbe \\ \hline


\end{tabular}
\end{frame}

\end{document}
