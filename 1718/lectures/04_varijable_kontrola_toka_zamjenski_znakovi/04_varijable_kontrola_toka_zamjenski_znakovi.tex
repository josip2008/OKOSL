	% sredit sadrzaj - ovo ne lici ni na sto
% pogreske - slajd preusmjeravanje u datoteku - ide u /tmp/test
% nezgodno ispisuje \ u \shell{}, mozda verbatim

\documentclass{beamer}

\usepackage[english]{babel}
\usepackage[utf8]{inputenc}
\usepackage{listings}
\usepackage{datetime}
\usepackage{graphics}
\usepackage{fancybox}
\usepackage{color}
\usepackage[normalem]{ulem}
\usepackage{tikz}
\usepackage{verbatim}
\usetikzlibrary{shapes,arrows,positioning}
\usetheme{CambridgeUS}
\usecolortheme{seagull}
% Changing of bullet foreground color not possible if {itemize item}[ball]
\DefineNamedColor{named}{Purple}{cmyk}{0.52,0.97,0,0.55}
\setbeamertemplate{itemize item}[triangle]
\setbeamercolor{title}{fg=Purple}
\setbeamercolor{frametitle}{fg=Purple}
\setbeamercolor{itemize item}{fg=Purple}
\setbeamercolor{section number projected}{bg=Purple,fg=white}
\setbeamercolor{subsection number projected}{bg=Purple}

\renewcommand{\dateseparator}{.}
\newcommand{\todayiso}{\twodigit\day \dateseparator \twodigit\month \dateseparator \the\year}
\newcommand{\shell}[1]{\texttt{#1}}
\lstset{basicstyle=\ttfamily,
	showstringspaces=false,
	commentstyle=\color{red},
	keywordstyle=\color{blue}
}

\title{Osnove korištenja operacijskog sustava Linux}
\subtitle{04. Varijable i kontrola toka. Zamjenski znakovi}
\author[Mateo Stjepanović]{Mateo Stjepanović\\{\small Nositelj: dr. sc. Stjepan Groš}}
\institute[FER]{Sveučilište u Zagrebu \\
				Fakultet elektrotehnike i računarstva}
				
\date{\todayiso}

\begin{document}
    %\beamerdefaultoverlayspecification{<+->}
{
\setbeamertemplate{headline}[] % still there but empty
\setbeamertemplate{footline}{}

\begin{frame}
\maketitle
\end{frame}
}

\begin{frame}
\frametitle{Sadržaj}
\tableofcontents
\end{frame}

\section{Varijable}
\begin{frame}[t]
\frametitle{Varijable}
\begin{itemize}
	\item Varijable se mogu koristiti kao u bilo kojem drugom jeziku
	\item Nije potrebno deklarirati varijablu (sličnost s Pythonom)
 	\item Svaka je varijabla implicitno tipa \textit{string}
	\item Dvije vrste varijabli
  	\begin{itemize}
  		\item Globalne varijable (Vidljivo iz više shellova)
  		\item Lokalne varijable (Vidljive samo u lokalnom shellu)
  	\end{itemize}
  
	\item Mogućnosti rada s varijablama
 	\begin{itemize}
 		\item Spremanje vrijednosti u varijablu
		\item Čitanje vrijednosti iz varijable
	\end{itemize}
\end{itemize}
\end{frame}



\begin{frame}[t]
\frametitle{Primjer postavljanja i čitanja varijabli}
\begin{itemize}
	\item Pridruživanje je jednostavnog oblika: varijabla=vrijednost
		 \begin{itemize}
		 	\item hello="Hello World"
		 	\item num=2
		 	\item VAŽNO: razmaka ne smije biti ni s jedne strane znaka jednakosti
		 \end{itemize}
	 \item varijabla=\$(\shell{naredba}) - preusmjeravanje izlaza naredbe u varijablu
	 \item \shell{export} varijabla - pretvaranje varijable u globalnu za \textbf{child} procese
	 \item \shell{echo} \$varijabla
	 	\begin{itemize}
	 		\item \shell{echo}  \$hello
	 		\item \shell{echo}  \$num
	 	\end{itemize}
 	\item Lokalna varijabla unutar skripte - \shell{local}
 	\item Perzistentno globalne varijable treba spremiti na neki drugi način:
 	\begin{itemize}
 		\item \textasciitilde/.bashrc
 		\item \textasciitilde/.profile
 		\item /etc/environment
 	\end{itemize}

\end{itemize}

\end{frame}
\begin{lstlisting}[language=bash]
#!/bin/bash
HELLO=Hello 
function hello {
	local HELLO=World
	echo $HELLO
}
echo $HELLO
hello
echo $HELLO
\end{lstlisting}
\section{Uvjeti}
\begin{frame}
\frametitle{Uvjeti}
\begin{itemize}
	\item Prije ulaska u kontrolu toka potrebno je navesti uvjet kontrole
	\item Uvjeti (\textit{tests}) su označeni s \shell{[ ]}
	\item Vraćaju 0 ili 1 ovisno o tome zadovoljava li test zadani uvjet
	\begin{itemize}
		\item \shell{[ 1 -eq 1 ]}
	\end{itemize}
	\item NB: potreban razmak između zagrada i uvjeta
	\item \shell{echo \$?} ispisuje rezultat testa
	\begin{itemize}
		\item ako je vraćena 0, to znači True (sjetite se C-ovskog \textit{return 0;})
	\end{itemize}
\end{itemize}
\end{frame}

\section{Operatori usporedbe}
\begin{frame}
\frametitle{Operatori usporedbe}
\begin{itemize}
	\item Operatori određuju uvjete kontrole toka
	\item Vrste operatora:
	\begin{itemize}
		\item Operatori integera
		\item Operatori stringa
		\item Operatori datoteka		
	\end{itemize}
\end{itemize}
\end{frame}

\begin{frame}
	\begin{columns}[onlytextwidth]
		\begin{column}{0.33\textwidth}
			\centering
			\textbf{Operatori integera} \\
			\begin{itemize}
				\item -lt (manje od)
				\item -gt (veće od)
				\item -le (manje od ili jednako)
				\item -ge (veće od ili jednako)
				\item -eq (jednako)
				\item -ne (različito)
			\end{itemize}
		\end{column}
		\begin{column}{0.33\textwidth}
			\centering
			\textbf{Operatori stringa} \\
			\begin{itemize}
				\item \textless (manje od)
				\item \textgreater (veće od)
				\item = (jednako)
				\item != (različito)
				\item -n (vraća 0 ako niz ima duljinu)
				\item -z (vraća 0 ako je niz prazan)
			\end{itemize}
		\end{column}
		\begin{column}{0.33\textwidth}
			\centering
			\textbf{Operatori datoteka} \\
			\begin{itemize}
				\item -e (datoteka postoji)
				\item -s (datoteka ima veličinu veću od 0)
				\item -d (datoteka je direktorij)
			\end{itemize}
		\end{column}
	\end{columns}
\end{frame}


\section{Kontrola toka}
\begin{frame}
\frametitle{Kontrola toka}
\begin{itemize}
	\item Sada se možemo upustiti u kontrolu toka
	\item Tri naredbe za kontrolu toka:
	\begin{itemize}
		\item \shell{if}
		\item \shell{else}
		\item \shell{elif}
	\end{itemize}
	\item Znajući gore navedene operatore i testove idemo prikazati primjere kontrole toka
\end{itemize}
\end{frame}
\begin{itemize}
	\item IF - struktura i primjer
\end{itemize}
	\begin{lstlisting}[language=bash]
#!/bin/bash
HELLO="hello"
if [ $HELLO = "hello" ]
then
	echo "$HELLO world"
fi
	\end{lstlisting}
\begin{itemize}
	\item ELSE - struktura i primjer
\end{itemize}
	
	\begin{lstlisting}[language=bash]
#! /bin/bash
if [ -e datoteka.txt ]
then
	cat datoteka.txt
else
	echo "Datoteka ne postoji!"
fi
	\end{lstlisting}
\begin{itemize}
	\item ELIF - struktura i primjer
\end{itemize}
	
	\begin{lstlisting}[language=bash]
#! /bin/bash
pom="kul"
if [ $pom = "kul" ]
then
	echo "Linux je $pom"
elif [ $pom = "bezveze" ]
then 
	echo "Linux masterrace"
else
	echo "Linux je OS intelektualne elite"
fi
	\end{lstlisting}


\section{Petlje}
\begin{frame}
\frametitle{Petlje}
	\begin{itemize}
		\item Bash, naravno, podržava i petlje
		\item \shell{while} i \shell{for} petlje konceptualno identične analozima iz drugih jezika
		\item Uz njih, postoje još i \shell{until} i \shell{select}

		\begin{itemize}
			\item \shell{for}: iterira po nekoj listi i \underline{\textbf{za}} svaki član liste radi nešto
			\item \shell{while}: izvršava se \underline{\textbf{sve dok}} je neki uvjet ispunjen
			\item \shell{until}: izvršava se \underline{\textbf{sve dok}} se neki uvjet \underline{\textbf{ne ispuni}}
			\item \shell{select}: generira jednostavan \underline{\textbf{izbornik}} na temelju predane liste
		\end{itemize}
	\end{itemize}
\end{frame}

\begin{frame}[fragile]
	\begin{itemize}
		\item \underline{FOR - struktura i primjer}
		\item "C-ovski" \shell{for}:
	\end{itemize}
	\begin{lstlisting}[language=bash]
echo "Brojevi od 42 do 27 djeljivi s 3: "
for (( i=42; i>=27; i=i-3)) do
	echo $i
done
	\end{lstlisting}

	\begin{itemize}
		\item \underline{"pythonovski" for (suštinski foreach):}
	\end{itemize}
	\begin{lstlisting}[language=bash]
names='Marvin Shooty Hactar Magrathea'
for name in $names; do
	echo $name;
done
	\end{lstlisting}
\end{frame}

\begin{frame}[fragile]
	\begin{itemize}
		\item \underline{WHILE - struktura i primjer}
	\end{itemize}
	\begin{lstlisting}[language=bash]
cnt=0
while [ $cnt -le 10 ]; do
	echo $cnt;
	((cnt++)); # expression expansion
done
	\end{lstlisting}

	\begin{itemize}
		\item \underline{obrada datoteke liniju po liniju}
	\end{itemize}
	\begin{lstlisting}[language=bash]
while read line; do
	echo $line; # ispisi liniju
	echo $line | rev; # obrni liniju
done
	\end{lstlisting}
\end{frame}


\begin{frame}[fragile]
	\begin{itemize}
		\item \underline{UNTIL - struktura i primjer}
	\end{itemize}
	\begin{lstlisting}[language=bash]
cnt=15
until [ $cnt -lt 10 ]; do
	echo $cnt;
	((cnt--));
done
	\end{lstlisting}

	\begin{itemize}
		\item \underline{SELECT - struktura i primjer}
	\end{itemize}
	\begin{lstlisting}[language=bash]
distros='Ubuntu Arch CentOS Gentoo Quit'
select distro in $distros; do
	if [ $distro == 'Quit' ]; then
		break;
	fi;
	echo Tvoja je omiljena distribucija $distro;
done
echo "Bilo mi je drago"
	\end{lstlisting}
\end{frame}


\section{Zamjenski znakovi}
\begin{frame}[t]
\frametitle{Zamjenski znakovi}
\begin{itemize}
  \item engl. wildcards
  \item Koriste se za brzo i efikasno pretraživanje i izvršavanje naredbi
\end{itemize}
\begin{itemize}
	\item Mogu se koristiti kod svih naredbi koje 
	prihvaćaju datoteke ili direktorije kao argument!
	\begin{itemize}
		\item Ljuska \textbf{prije} pokretanja naredbi uklanja zamjenske znakove, pronalazi sve datoteke koje odgovaraju izrazu i postavlja ih umjesto zamjenskog izraza kao da su direktno uneseni
		\item Potom pokreće naredbu koja ne dobiva nikakve informacije o zamjenskim znakovima!
	\end{itemize}
\end{itemize}
\end{frame}

\begin{frame}[t]
\frametitle{Zamjenski znakovi}
\begin{itemize}
	\item[] \textbf{Osnovni zamjenski znakovi}
	\item[] \begin{tabular}{c l}
		\shell{?}   & Odgovara točno jednom znaku\\
		\shell{*}   & Odgovara bilo kojem broju znakova (ili nijednom)\\
		\shell{[xyz]}  & Odgovara točno jednom znaku iz skupa xyz
	\end{tabular}
\end{itemize}
\vfill
\begin{itemize}
	\item \textbf{Primjer 1}
	\item[] Izlistajte sve datoteke čiji naziv počinje s b i nalaze se u \shell{/bin} direktoriju
	\begin{itemize}
		\item[] \shell{\$ ls -l /bin/b*}
	\end{itemize}
\end{itemize}
\vfill
\begin{itemize}
	\item \textbf{Primjer 2}
	\item[] Izlistati sve naredbe u \shell{/bin} direktoriju koje se sastoje od točno dva znaka
	\begin{itemize}
		\item[] \shell{\$ ls -l /bin/??}
	\end{itemize}
\end{itemize}
\end{frame}

\begin{frame}[t]
\frametitle{Zamjenski znakovi}
\begin{itemize}
	\item \textbf{Primjer 3}
	\item[] Izlistati sve datoteke u \shell{/bin} direktoriju koje završavaju slovom \shell{d}
  \begin{itemize}
    \item[] \shell{\$ ls -l /bin/*d}
  \end{itemize}
\end{itemize}
\vfill
\begin{itemize}
	\item \textbf{Primjer 4}
	\item[] Ispisati sve datoteke u \shell{/bin} direktoriju koje započinju s a, b ili c
  \begin{itemize}
    \item Jedna mogućnost
    \item[] \shell{\$ ls -l /bin/a* /bin/b* /bin/c*}
    \item Kraće
    \item[] \shell{\$ ls -l /bin/[abc]*}
  \end{itemize}
\end{itemize}
\vfill
\end{frame}

\begin{frame}[t]
\frametitle{Zamjenski znakovi}
\begin{itemize}
	\item \textbf{Primjer 5}
	\item[] Izlistati sve datoteke u \shell{/bin} direktoriju koje u sebi 
        sadrže barem jednu znamenku
  \begin{itemize}
    \item[] \shell{\$ ls -l /bin/*[0123456789]*}
    \item Efikasnije: Zadavanjem \textbf{raspona znakova}
    \item[] \shell{\$ ls -l /bin/*[0-9]*}
  \end{itemize}
\end{itemize}
\vfill
\end{frame}

\begin{frame}[t]
\frametitle{Zamjenski znakovi}
\framesubtitle{Invertiranje skupa}
\begin{itemize}
  \item Korištenjem znaka \textasciicircum{} moguće je invertirati skup 
        znakova u zagradi 
\end{itemize}
\vfill
\begin{itemize}
\item \textbf{Primjer 6}
  \item[] Izlistati sve datoteke u direktoriju \shell{/usr/bin} koje ne 
        započinju sa malim slovom abecede
  \begin{itemize}
    \item[] \shell{\$ ls -l /usr/bin/[\textasciicircum{}a-z]*}
  \end{itemize}
\end{itemize}
\vfill
\begin{itemize}
  \item Ako znakove \textasciicircum{} i \shell{-} trebamo koristiti kao dio traženog skupa znakova tada \textasciicircum{} ne smije biti naveden na prvom mjestu u grupi, a \shell{-} mora biti naveden kao prvi ili zadnji znak
\end{itemize}
\end{frame}

\begin{frame}[t]
\frametitle{Zamjenski znakovi}
\framesubtitle{Isključenje značenja posebnih znakova}
\begin{itemize}
  \item Ponekad ne želimo posebno značenje zamjenskih znakova
  \begin{itemize}
    \item Što ako baš imamo datoteku koja se zove \shell{*} ?
    \item U tom slučaju upotrebljavamo navodnike ili znak 
          \shell{\char`\\}
  \end{itemize}
\end{itemize}
\small
  \begin{itemize}
    \item[] \shell{\$ ls -l "/bin/b*"}
    \item[] \shell{\$ ls -l /bin/b\char`\\{}*}
  \end{itemize}
\end{frame}

\begin{frame}
\frametitle{Literatura}
\begin{itemize}
	\item[]
	\item[] \small\url{http://blog.tgrrtt.com/bash-101}
	\item[] \small\url{http://linuxcommand.org/lc3_wss0110.php}
	\item[] \small\url{https://www.cyberciti.biz/faq/set-environment-variable-linux/}
	\item[] \small\url{https://ryanstutorials.net/bash-scripting-tutorial/bash-variables.php}
	\item[] \small\url{http://tldp.org/LDP/Bash-Beginners-Guide/html/sect_03_02.html} 
	\item[] \small\url{http://tldp.org/HOWTO/Bash-Prog-Intro-HOWTO-5.html}
\end{itemize}
\end{frame}
\end{document}
