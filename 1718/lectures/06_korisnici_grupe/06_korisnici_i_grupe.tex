\documentclass[table,usenames,dvipsnames] {beamer}

\usepackage[english]{babel}
\usepackage[utf8]{inputenc}
\usepackage{listings}
\usepackage{datetime}
\usepackage{graphics}
\usepackage{fancybox}
\usepackage{color}
\usepackage{courier}
\usepackage[normalem]{ulem}
\usepackage{tikz}
\usepackage{verbatim}
\usepackage{multirow}
\usepackage{fancyvrb}
\usetikzlibrary{shapes,arrows}
\usetheme{CambridgeUS}
\usecolortheme{seagull}
% Changing of bullet foreground color not possible if {itemize item}[ball]
\DefineNamedColor{named}{Purple}{cmyk}{0.52,0.97,0,0.55}
\setbeamertemplate{itemize item}[triangle]
\setbeamercolor{title}{fg=Purple}
\setbeamercolor{frametitle}{fg=Purple}
\setbeamercolor{itemize item}{fg=Purple}
\setbeamercolor{section number projected}{bg=Purple,fg=white}
\setbeamercolor{subsection number projected}{bg=Purple}

\renewcommand{\dateseparator}{.}
\newcommand{\todayiso}{\twodigit\day \dateseparator \twodigit\month \dateseparator \the\year}
\newcommand{\shell}[1]{\texttt{#1}}

\title{Osnove korištenja operacijskog sustava Linux}
\subtitle{05. Korisnici i grupe, vlasništvo i dozvole}
\author[Lucija Petricioli, Josip Žuljević, Dominik Barbarić]{Lucija Petricioli, Josip Žuljević, Dominik Barbarić\\{\small Nositelj: doc. dr. sc. Stjepan Groš}}
\institute[FER]{Sveučilište u Zagrebu \\
				Fakultet elektrotehnike i računarstva}
				
\date{\todayiso}

\begin{document}
    %\beamerdefaultoverlayspecification{<+->}
{
\setbeamertemplate{headline}[] % still there but empty
\setbeamertemplate{footline}{}

\begin{frame}
\maketitle
\end{frame}
}

\begin{frame}
\frametitle{Sadržaj}
\tableofcontents
\end{frame}

\section{Terminal i višekorisnički sustav}
\begin{frame}[t]
\frametitle{Osnovni pojmovi (1)}
\begin{itemize}
  \item Linux je višekorisnički operacijski sustav
  \item Uloge višekorisničkog rada u OS-u:
  \begin{itemize}
    \item Zaštita privatnosti
    \item Specifične postavke i podaci
    \item Sprečavanje zlouporabe
    \item Pravedna raspodjela resursa
  \end{itemize}
\end{itemize}
\end{frame}

\begin{frame}[t]
\frametitle{Osnovni pojmovi (2)}
\begin{itemize}
	\item Terminal -- U/I naprava za komunikaciju korisnika s računalom
	\item Nekada fizički uređaj, danas programski emulatori
	\item Omogćuju korisniku prikaz \textit{ljuske} - npr. \shell{bash}
  \item Prijava na sustav
  \begin{itemize}
  	\item Prijava lozinkom ili drugim vjerodajnicama
  	\item Odmah po prijavi u sustav korisnik je smješten u svoj matični direktorij
  \end{itemize}
  \item Odjava iz sustava
  \begin{itemize}
  	\item Iz \shell{bash} ljuske ostvaruje se:
    \begin{itemize}
      \item naredbom \shell{logout}
      \item \shell{exit} 
      \item kombinacijom \shell{CTRL+D} - slanje signala \shell{SIGQUIT}
    \end{itemize}
  	\item Terminal otvara upit za prijavu novog korisnika
  \end{itemize}
\end{itemize}
\end{frame}

\begin{frame}[t]
\frametitle{Osnovni pojmovi (3)}
  \begin{itemize}
  	\item Terminali su predstavljeni datotekama uređaja:
    \item[] \shell{tty0, tty1, tty2, \ldots}
    \begin{itemize}
    	\item Terminalima upravlja upravljački program - \shell{getty}
    	\item Kod modernih - virtualnih - terminala korisnik može s istog mjesta koristiti više terminala
	  \item Između terminala se prebacuje sa Ctrl+Alt+F1\ldots F7
    \item Ctrl+Alt+F7 vraća u konzolu s grafičkim sučeljem (npr. X)
    \end{itemize}
    \item[] \shell{pts/N}
    \begin{itemize}
      \item Označavaju pseudoterminale - programski emulirane
      \item Oni su "terminal" na koji se danas najčešće misli
      \item[] Npr. \shell{gnome-terminal}
    \end{itemize}
  \end{itemize}
\end{frame}

\section{Baza korisnika}
\begin{frame}[t]
	\frametitle{Baza passwd (1)} 
	\begin{itemize}
		\item Temeljna datoteka s korisnicima je \shell{/etc/passwd}
		\begin{itemize} 
			\item Povezuje korisničko ime i UID
			\item Nekada je u njoj bila i lozinka
			\item Vrlo loše sa sigurnosne strane - ne može se zabraniti njeno čitanje jer mnoštvo aplikacija ovisi o podacima u toj datoteci
		\end{itemize}
		\item Sadrži jedan zapis po liniji oblika
		\begin{itemize} {\footnotesize
			\item[] \hspace{-2em} Korisničko ime:Lozinka:UID:GID (primarna grupa):Info:Matični direktorij:Korisnička ljuska
			\item[] \shell{root:x:0:0:root:/root:/bin/bash}
		} \end{itemize}
		\item Uređivanje naredbom \shell{vipw}
    \begin{itemize}
      \item Zaštita od paralelnog uređivanja
      \item Osnovno parsiranje i sintaksna provjera
    \end{itemize}
	\end{itemize}
\end{frame}

\begin{frame}[t]
	\frametitle{Baza passwd (2)}
	\begin{itemize}
		\item Korisnici navedeni u passwd datoteci ne moraju biti (i uglavnom nisu) interaktivni korisnici 
    \item Neke korisnike koriste servisi koji ne trebaju izravno logiranje u ljusku.
	\end{itemize}
	\begin{itemize}
		\item Matični direktorij korisnika ne mora biti zadan ako se ne radi o interaktivnom korisniku
		\item Ljuska određuje koji se program koristi prilikom prijave korisnika
		\begin{itemize}
	 		\item[] \shell{/bin/bash} - Moguća vrijednost za interaktivnog korisnika
			\item[] \shell{/bin/false} - Moguća vrijednost za korisnika bez mogućnosti prijave na sustav
		\end{itemize}
	\end{itemize}
\end{frame}

\begin{frame}[t]
	\frametitle{Baza shadow}
	\begin{itemize}
		\item Ako u passwd bazi na mjestu lozinke stoji \shell{x} tada se sigurnosni podaci o korisniku nalaze u datoteci \shell{/etc/shadow}
		\begin{itemize}
			\item Sadrži kriptirane lozinke, te dodatne podatke o njihovom trajanju
			\item Čitljiva je isključivo root korisniku
		\end{itemize}
	\end{itemize}
	\begin{itemize}
		\item Sadrži jedan zapis po liniji oblika
		\begin{itemize} {\footnotesize
			\item[] \hspace*{-3em} Korisničko ime:Lozinka:Polja s dodatnim podacima
			\item[] \shell{root:T3RqrzxU1MAH3F3wtuQu/:13284:0:99999:7:::}
		} \end{itemize}
	\end{itemize}
\end{frame}

\begin{frame}[t]
\frametitle{Naredba \shell{who}}
\begin{itemize}
  \item Naredba može prikazati podatke o korisniku
  \item Primjer ispisa
  \begin{itemize}
    \item[] \shell{\$ who}
    \begin{table}[h]
    \begin{flushleft}
    \shell{
    \begin{tabular}{l l l l l}
      cetko & tty7 & 2010-11-11 & 12:01 & (:0)  \\
      cetko & pts/0 & 2010-11-11 & 17:08 & (:0) \\
      cetko & pts/1 & 2010-11-11 & 17:08 & (:0) \\ 
      cetko & pts/2 & 2010-11-11 & 17:12 & (:0)  
    \end{tabular} }
    \end{flushleft}
    \end{table}
  \end{itemize}
  \item Poseban oblik naredbe \shell{who} je \shell{who am i}
  \begin{itemize}
    \item Ispisuje tko je trenutni korisnik na trenutnom terminalu
  \end{itemize}
  \item Varijanta te naredbe je \shell{whoami}
  \begin{itemize}
    \item Ispisuje samo korisničko ime
  \end{itemize}
\end{itemize}
\end{frame}

\begin{frame}[t]
\frametitle{Naredba \shell{finger}}
\begin{itemize}
  \item Drugi način prikaza trenutno aktivnih korisnika
  \item Prikazuje trenutno logirane korisnike, ili prikazuje 
          detaljnije podatke o nekom korisniku
\end{itemize}
\begin{itemize}
   \item Prikazuje dodatne podatke
   \begin{itemize}
   	\item Iz \shell{Info} polja u passwd bazi
   	\item Čita ih iz datoteka \shell{.project} i \shell{.plan} u matičnom direktoriju
   \end{itemize}
   \item Ako joj zadamo parametar pretražuje korisnika
   \begin{itemize}
   	\item Pretraživanje se obavlja po korisničkom imenu i pravom imenu
   \end{itemize}
\end{itemize}
\end{frame}

\begin{frame}[t]
\frametitle{Naredba \shell{w}}
\begin{itemize}
  \item Primjer ispisa
    \begin{table}[h]\footnotesize
    \begin{flushleft}
    \shell{
    \begin{tabular}{l l l l l l l l}
      USER & TTY & FROM & LOGIN@ & IDLE & JCPU & PCPU & WHAT  \\
      cetko & tty7 & :0 & 12:01 & 5:32m & 3:45 & 9.67s & awesome \\
      cetko & pts/0 & :0 & 17:29 & 3:21 & 0.33s & 0.33s & bash \\
      cetko & pts/1 & :0 & 7:31 & 1:06 & 0.33s & 0.33s & bash \\
      cetko & pts/5 & :0 & 17:23 & 0.00s & 0.32s & 0.00s & w
    \end{tabular} }
    \end{flushleft}
    \end{table}
\end{itemize}
\end{frame}

\begin{frame}[t]
\frametitle{root} 
\begin{itemize}
  \item Operacijski sustav korisnike identificira preko jedinstvenog identifikatora
  \begin{itemize}
	  \item[] \textbf{UID (User ID)}
  \end{itemize}
\end{itemize}
\begin{itemize}
  \item Jedan korisnik se posebno tretira 
  \begin{itemize}
    \item[] \bf{root} \hspace{2em} UID=0
  \end{itemize}
\end{itemize}
\begin{itemize}
  \item \textbf{root može sve!}
  \begin{itemize}
    \item \textbf{Nije preporučljivo ulogiravati se i/ili raditi kao root!}
    \item Raditi kao običan korisnik pa tek kad je nužno prebaciti se na 
          root korisnika - ako je ikako moguće, kroz \shell{sudo}
  \end{itemize}
\end{itemize}
\end{frame}

\begin{frame}[fragile]
\frametitle{sudo}
\begin{itemize}
  \item Sučelje za privremeno dobivanje administrativnih ovlasti
  \item sudo mogu izvršiti svi korisnici prema dozvolama definiranima u datoteci
  \item[] {\small\shell{/etc/sudoers}}
  \item Uređivanje naredbom \shell{visudo}, iz istih razloga kao i \shell{vipw}
\end{itemize}
\vfill
\footnotesize
\begin{Verbatim}
dino     ALL = (ALL) ALL
dominik  marvin,magrathea = (dino) /bin/dd

%kset    ALL = NOPASSWD: /sbin/umount /media/cdrom0
\end{Verbatim}
\end{frame}


\begin{frame}[t]
\frametitle{Mijenjanje korisnika} 
\begin{itemize}
  \item Vrlo bitna naredba \shell{su} (engl. \emph{switch user})
  \item Dva bitna oblika naredbe
  \begin{itemize}
    \item \shell{su <korisnicko ime>} - zadržava svojstva okoline (varijable i slično)
    \item \shell{su - <korisnicko ime>} - stvara novu okolinu, svojstvenu korisniku
  \end{itemize}
  \item Bez argumenata mijenja korisnika u \shell{root}
\end{itemize}
\end{frame}

\section{Grupe}
\begin{frame}[t]
\frametitle{Grupe (1)}
\begin{itemize}
  \item Korisnici se grupiraju u korisničke grupe
  \begin{itemize}
  	\item Administracija korisnika
  	\item Dijeljenje podataka
  	\item Zajedničke dozvole
  \end{itemize}
  \item Svaki korisnik ima
  \begin{itemize}
    \item[] \textbf{Primarnu grupu}
    \begin{itemize}
      \item Zapisana u datoteci \shell{etc/passwd} 
    \end{itemize}
    \item[] \textbf{Sekundarne grupe}
    \begin{itemize}
      \item  Sve grupe kojima korisnik pripada
    \end{itemize}
  \end{itemize}
\end{itemize}
\end{frame}

\begin{frame}[t]
	\frametitle{Grupe (2)}
	\begin{itemize}
		\item Slično kao i za korisnike za grupe se koristi groups baza u datoteci \shell{/etc/group}
		\item Sadrži jedan zapis po liniji oblika
		\begin{itemize} {\footnotesize
			\item[] \hspace{-2em} Ime grupe:Lozinka:GID:Popis korisnika
			\item[] \shell{cdrom:x:24:linux,dominik,dino}
		} \end{itemize}
	  \item Grupe također imaju posebnu datoteku za lozinke \shell{/etc/gshadow}
	\end{itemize}
	\begin{itemize}
	  \item Operacijski sustav i s grupama radi preko jedinstvenog identifikatora
	  \begin{itemize}
		\item[] \textbf{GID (Group ID)}
	  \end{itemize}
	  \item Naredbom \shell{id} saznajemo sve grupe u koje korisnik pripada
	  \begin{itemize}
		\item[] \shell{uid=1000(user) gid=1000(user) groups=1000(user),4(adm)...}
	  \end{itemize}
      \item Privremena prijava u druge grupe naredbom \shell{newgrp}
	 \end{itemize}
\end{frame}


\section{Upravljanje korisnicima}
\begin{frame}[t]
\frametitle{Upravljanje korisnicima}
\begin{itemize}
  \item Osnovne operacije s korisnicima
  \begin{itemize}
    \item Dodavanje novog korisnika
    \begin{itemize}
      \item[] \shell{adduser}
    \end{itemize}
    \item Promjena lozinke korisnika
    \begin{itemize}
      \item[] \shell{passwd}
    \end{itemize}
    \item Promjena podataka o korisniku
    \begin{itemize}
      \item[] \shell{usermod}
    \end{itemize}
    \item Uklanjanje korisnika
    \begin{itemize}
      \item[] \shell{deluser}
    \end{itemize}
  \end{itemize}
\end{itemize}
\begin{itemize}
	\item Analogne naredbe postoje i za grupe
	\begin{itemize}
		\item[] \shell{groupadd, groupmod, groupdel}
	\end{itemize}
\end{itemize}
\end{frame}

\begin{frame}[t]
\frametitle{Upravljanje korisnicima}
\begin{itemize}
  \item Stvaranje novog korisnika
  \begin{itemize}
    \item[] \shell{\$ adduser <korisnik>}
  \end{itemize}
  \item Dodavanje korisnika postojećoj grupi
  \begin{itemize}
  	\item[] \shell{\$ usermod -aG <grupa> <korisnik>}
    \item[ili] \shell{\$ adduser <korisnik> <grupa>}
  \end{itemize}
  \item Stvaranje nove grupe
  \begin{itemize}
    \item[] \shell{\$ addgroup <grupa>}
    \item[ili] \shell{\$ adduser --group <grupa>}
  \end{itemize}
\end{itemize}
\end{frame}

\begin{frame}[t]
\frametitle{Promjena podataka o korisniku}
\begin{itemize}
  \item Promjena podataka o korisniku
  \begin{itemize}
    \item Mogu se mjenjati svi podaci
    \begin{itemize}
      \item[] \shell{usermod <opcije> <username>}
    \end{itemize}
    \item Promjena ljuske, opcija \shell{-s <shell>}
    \item Promjena matičnog direktorija, opcija \shell{-d <dir>}
  \end{itemize}
\end{itemize}
\begin{itemize}
  \item Ljuska korisnika može se promijeniti i  naredbom \shell{chsh}
  \item Naredba \shell{chfn} mijenja dodatne podatke o korisnicima
  \begin{itemize}
  	\item[] Finger podaci - \shell{Info} polje
  \end{itemize}
  \item Lozinka se mijenja naredbom \shell{passwd}
\end{itemize}
\end{frame}

\begin{frame}[t]
\frametitle{Upravljanje korisnicima}
\begin{itemize}
  \item Brisanje kreiranog korisnika
  \begin{itemize}
    \item[] \shell{\$ deluser <korisnik>}
  \end{itemize}
  \item Brisanje korisnika iz grupe
  \begin{itemize}
    \item[] \shell{\$ deluser <korisnik> <grupa>}
  \end{itemize}
  \item Brisanje grupe
  \begin{itemize}
  	\item[] \shell{\$ delgroup <grupa>}
    \item[ili] \shell{\$ deluser --group <grupa>}
  \end{itemize}
\end{itemize}
\end{frame}

\begin{frame}[t]
\frametitle{Upravljanje korisnicima}
\begin{itemize}
  \item Kod stvaranja korisnika se može definirati lokacija matičnog 
        direktorija i njegovo brisanje zajedno sa korisnikom
  \item Navedene naredbe su sučelja drugih naredbi
  \begin{itemize}
    \item[] \shell{adduser} $\Rightarrow$ \shell{useradd}
    \item[] \shell{deluser} $\Rightarrow$ \shell{userdel}
    \item[] \shell{addgroup} $\Rightarrow$ \shell{groupadd}
    \item[] \shell{delgroup} $\Rightarrow$ \shell{groupdel}
  \end{itemize}
  \item Sve prethodne akcije se mogu napraviti i navedenim naredbama
\end{itemize}
\end{frame}

\begin{frame}[t]
\frametitle{Upravljanje korisnicima}
\begin{itemize}
  \item Ako kod stvaranja korisnika nisu definirani parametri, koriste se 
        postavke u \shell{/etc/adduser.conf}
  \item U matičnom direktoriju se stvaraju predefinirane datoteke
  \begin{itemize}
     \item Raspored početnih datoteka je definiran u direktoriju \shell{/etc/skel} (engl. \emph{skeleton})
  \end{itemize}
\end{itemize}
\begin{itemize}
  \item Zadatak
  \begin{itemize}
    \item Proučiti opcije u datoteci \shell{/etc/adduser.conf}
    \item Izlistati direktorij \shell{/etc/skel} i matični direktorij
  \end{itemize}
\end{itemize}
\end{frame}

\begin{frame}[t]
	\frametitle{Naredbe}
	\begin{table}[h]
		%  \rowcolors{2}{White}{LightGray}
		\begin{tabular}{|c|l|}
			\hline
			\rowcolor{BlueViolet!20}Naredba & Opis \\
			\hline
			Ctrl+D & odjava iz terminala \\
			\hline
			logout & odjava iz terminala \\
			\hline
			who & prikazuje podatke o korisniku \\
			\hline
			who am i & ispisuje korisnika u trenutnom terminalu \\
			\hline
			whoami & ispisuje isključivo korisničko ime korisnika u terminalu \\
			\hline
			finger & ispisuje trenutno aktivne korisnike \\
			\hline
			su & izmjena korisnika \\
			\hline
			newgrp & prijava u drugu grupu \\
			\hline
			usermod & izmjena podataka o korisniku \\
			\hline
			passwd & promjena korisničke lozinke \\
			\hline
		\end{tabular}
	\end{table}
\end{frame}

\section{Dozvole}
\begin{frame}[t]
\frametitle{Dozvole (1)}
\begin{itemize}
  \item Naredba \shell{ls -l} ispisuje informacije o vlasnicima i dozvolama objekta
  {\small \item[] \shell{\$ ls -l datoteka.txt}
  \item[] \shell{-\textbf{rw-r--r-- 1 pero users} 0 Jan  4 23:19 datoteka.txt}}
  \vfill
    \item Objekt je vlasništvo korisnika i grupe
    \begin{itemize}
    \item Drugo polje označava vlasnika - korisnika \hfill (\shell{pero}) \hfill \,
    \item Treće polje označava vlasnika - grupu \hfill (\shell{users}) \hfill \,
    \end{itemize}
    \item Prvo polje u prvom bitu sadrži oznaku tipa datoteke, a ostalih 9 bitova se nazivaju \textbf{mode} objekta
  \end{itemize}
\end{frame}

\begin{frame}[t]
\frametitle{Dozvole (2)}
\begin{itemize}
  \item \textbf{mode} definira dozvoljene operacije na svakom objektu
  \vspace{1em}
  \item Devet bitova dijele se u tri grupe od koji svaka čini jedan troznamenkasti binarni broj
  \item Svaki troznamenkasti binarni broj se može prikazati jednom oktalnom znamenkom
\end{itemize}
\begin{itemize}
  \item Svaka oktalna znamenka modea predstavlja skup dozvola koje su dodijeljene sljedećim korisnicima objekta i to:
  \begin{itemize}
  \item Prva oktalna znamenka definira prava za vlasnika - korisnika \hfill \textbf{user}
    \item Druga oktalna znamenka definira prava za vlasnika - grupu \hfill \textbf{group}
    \item Treća oktalna znamenka definira prava za sve ostale \hfill \textbf{others}
  \end{itemize}
\end{itemize}
\end{frame}

\begin{frame}[t]
\frametitle{Dozvole (3)}
\begin{itemize}
  \item Značenja pojedinih bitova svake znamenke
    \begin{tabular}{l l}
     r & \textbf{read} - Dozvoljeno čitanje \\
     w & \textbf{write} - Dozvoljeno pisanje \\
     x & \textbf{execute} - Dozvoljeno izvršavanje / pretraživanje direktorija
    \end{tabular}
      \item Svaki pojedini bit može biti u stanju
      \begin{itemize}
        \item \textbf{uključen} - operacija dozvoljena
        \item \textbf{isključen} - operacija zabranjena
      \end{itemize}
\end{itemize}
\vfill
\begin{itemize}
  \item \textbf{Primjer 1}
  \begin{itemize}
    \item[] \shell{rwxr-xr-x} = $111101101_2$ = $755_8$
    \item[] \hspace{1em} \begin{tabular}{l c c c}
      & \textbf{r} & \textbf{w} & \textbf{x}\\
      \textbf{user} & + & + & +\\
      \textbf{group} & + & - & +\\
      \textbf{others} & + & - & +
    \end{tabular}
  \end{itemize}
\end{itemize}
\end{frame}

\begin{frame}[t]
  \frametitle{Dozvole (4)}
\begin{itemize}
  \item \textbf{Primjer 2}
  \begin{itemize}
      \item[] \shell{rw-r--r--} = $644$
      \item[] \hspace{1em} \begin{tabular}{l c c c}
        & \textbf{r} & \textbf{w} & \textbf{x}\\
        \textbf{user} & + & + & -\\
        \textbf{group} & + & - & -\\
        \textbf{others} & + & - & -
      \end{tabular}
  \end{itemize}
\end{itemize}
\vfill
\begin{itemize}
  \item \textbf{Primjer 3}
  \begin{itemize}
      \item[] \shell{r--r--r--} = $444$
      \item[] \hspace{1em} \begin{tabular}{l c c c}
        & \textbf{r} & \textbf{w} & \textbf{x}\\
        \textbf{user} & + & - & -\\
        \textbf{group} & + & - & -\\
        \textbf{others} & + & - & -
      \end{tabular}
  \end{itemize}
\end{itemize}
\end{frame}

\begin{frame}[t]
\frametitle{Promjena dozvola (1)}
\begin{itemize}
  \item Promjena modea obavlja se naredbom \shell{chmod}
  \begin{itemize}
    \item[] \shell{chmod <mode> <objekt>}
  \end{itemize}
  \item Mode se može zadati oktalno i simbolički
  \item Moguće jer rekurzivno mijenjati prava
  \begin{itemize}
    \item[] \shell{chmod -R <mode> <objekt>}
  \end{itemize}
  \vfill
    \item \textbf{Vlasnik} datoteke može bez obzira na trenutni mod
    \begin{itemize}
      \item promijeniti mode
      \item obrisati datoteku
    \end{itemize}
\end{itemize}
\end{frame}

\begin{frame}[t]
\frametitle{Promjena dozvola (2)}
\begin{itemize}
  \item \textbf{Primjer 4}
  \begin{itemize}
    \item[] \shell{chmod ugo=rwx file1}
    \item[] \hspace{1em} \begin{tabular}{l c c c}
      & \textbf{r} & \textbf{w} & \textbf{x}\\
      \textbf{user} & + & + & +\\
      \textbf{group} & + & + & +\\
      \textbf{others} & + & + & +
    \end{tabular}
    \item \vspace{1em} Alternativno:
    \item[] \shell{chmod a=rwx file1}
    \item[] \shell{chmod 777 file1}
  \end{itemize}
\end{itemize}
\begin{itemize}
  \item \textbf{Primjer 5}
    \begin{itemize}
      \item[] \shell{chmod u=rwx,go=rx file1 file2}
      \item[ili] \shell{chmod 755 file1 file2}
  \end{itemize}
\end{itemize}
\end{frame}


\begin{frame}[t]
  \frametitle{Promjena dozvola (3)}
\begin{itemize}
  \item \textbf{Primjer 6}
  \begin{itemize}
      \item[] \shell{chmod g+w file1 file2 file3}
      \item[] \hspace{1em} \begin{tabular}{l c c c}
      & \textbf{r} & \textbf{w} & \textbf{x}\\
      \textbf{user} & $\ast$ & $\ast$ & $\ast$\\
      \textbf{group} & $\ast$ & + & $\ast$\\
      \textbf{others} & $\ast$ & $\ast$ & $\ast$
    \end{tabular}
    \end{itemize}
  \item \textbf{Primjer 7}
  \begin{itemize}
      \item[] \shell{chmod -x file1} 
      \item[ili] \shell{chmod a-x file1}
      \item[] \hspace{1em} \begin{tabular}{l c c c}
            & \textbf{r} & \textbf{w} & \textbf{x}\\
            \textbf{user} & $\ast$ & $\ast$ & -\\
            \textbf{group} & $\ast$ & $\ast$ & -\\
            \textbf{others} & $\ast$ & $\ast$ & -
          \end{tabular}
    \end{itemize}
\end{itemize}
\end{frame}


\defverbatim[colored]\makeset{
  \begin{lstlisting}[language=bash, basicstyle=\footnotesize\ttfamily, showstringspaces=false, keywordstyle=\color{blue}]
#!/bin/bash
echo "Skripta je pokrenuta"
  \end{lstlisting}
}

\begin{frame}[fragile]
\frametitle{Izvršavanje datoteka}
\begin{itemize}
  \item Svaka datoteka na UNIX sustavu može biti izvršna (\textit{executable})
  \item Skripta se, tako, može izvršiti korištenjem zadanog interpretora
\end{itemize}
\begin{itemize}
  \item Postavljanjem \shell{x} dozvole svaka se datoteka može izvršiti izravnim pozivanjem
\end{itemize}
\vfill
\begin{block}{/home/linux/skripta.sh \hfill mode 755}
  \makeset
\end{block}
\vfill
\begin{Verbatim}[fontsize=\footnotesize]
~$ /home/linux/skripta.sh
Skripta je pokrenuta
~$ ./skripta.sh
Skripta je pokrenuta
\end{Verbatim}
\end{frame}


\begin{frame}
  \frametitle{Promjena dozvola (4)}
\begin{itemize}
  \item Naredba \shell{chmod} može prihvatiti poseban argument prilikom simboličkog zadavanja modea
\end{itemize}
\begin{itemize}
  \item[] \shell{X} (veliko X)
   \begin{itemize}
    \item Direktorijima postavlja \shell{x} dozvolu
    \item Ostalim datotekama ne mijenja mod
    \item Omogućuje listanje direktorija bez dodavanja dozvole za izvršavanje datoteka
    \item Koristan prilikom rekurzivne promjene modea:
    \item[] \hspace{1em} \shell{chmod -R a+X dir1}
   \end{itemize}
\end{itemize}
\end{frame}

\section{Posebne dozvole}
\begin{frame}[t]
  \frametitle{Posebne dozvole (1)}
  \framesubtitle{Sticky bit}
\textbf{Sticky bit} / Text mode
\begin{itemize}
  \item \textbf{Kod direktorija}
  \item[] Dozvoljava brisanje direktorija \textbf{samo} vlasniku i root korisniku
  \item \textbf{Kod datoteka}
  \item[] Nakon izvršavanja datoteke proces ostaje u memoriji
\end{itemize}
\begin{itemize}
  \item Simbolički se označava s velikim \shell{T} na mjestu \shell{x} dozvole za \textit{others} korisnike
  \item[] {\footnotesize \shell{-rwxr--r-T 1 pero users 0 Jan  4 23:21 datoteka.txt} }
  \item Ako \textit{others} ujedno ima i \shell{x} dozvolu tada se sticky bit označava s malim \shell{t}
\end{itemize}
\end{frame}

\begin{frame}[t]
  \frametitle{Posebne dozvole (2)}
  \framesubtitle{SUID i SGID}
\begin{itemize}
  \item Za razumijeti preostala dva posebna bita potrebno je shvatiti što se događa s dozvolama korisnika koji pokreće izvršnu datoteku
  \item Svaki proces se pokreće s UID i GID primarne grupe korisnika koji ga je pozvao. Pokrenuti proces ima sve ovlasti tog korisnika
\end{itemize}
\vfill
\textbf{Set user ID (SUID)} i \textbf{Set group ID (SGID)}
\begin{itemize}
  \item Postavljanjem ovih bitova u mode datoteke proces koji pokreće datoteku dobiva dozvole \textbf{vlasnika - korisnika} (SUID bit), odnosno \textbf{vlasnika - grupe} (SGID) izvršne datoteke
\end{itemize}
\end{frame}

\begin{frame}[t]
  \frametitle{Posebne dozvole (3)}
  \framesubtitle{SUID i SGID}
\begin{itemize}
  \item Simbolički se označava s velikim \shell{S} na mjestu \shell{x} dozvole za određenu grupu korisnika
  \item[] {\footnotesize \shell{-rwSr--r-x 1 pero users 0 Jan  4 23:21 datoteka.txt} \hfill SUID }
  \item[] {\footnotesize \shell{-rw-r-Sr-x 1 pero users 0 Jan  4 23:21 datoteka.txt} \hfill SGID }
\end{itemize}
\begin{itemize}
  \item Primijetite da SUID, odnosno SGID ne impliciraju \shell{x} dozvolu vlasnicima datoteke. U gornjem primjeru samo \textit{others} imaju pravo izvršiti datoteku i u tom trenutku će isti dobiti prava vlasnika.
  \item Ako vlasnik, \textit{user} ili \textit{group} ujedno ima i \shell{x} dozvolu tada se posebni bitovi označavaju s malim \shell{s}
\end{itemize}
\end{frame}

\begin{frame}
  \frametitle{Posebne dozvole (4)}
  \framesubtitle{Promjena dozvola}
\begin{itemize}
  \item Posebne dozvole se također mijenjaju naredbom \shell{chmod}
  \item Ispred uobičajene tri znamenke dodaje se još jedna čiji bitovi odgovaraju posebnim dozvolama
  \begin{itemize}
    \item \textbf{Prvi bit} - SUID
    \item \textbf{Drugi bit} - SGID
    \item \textbf{Treći bit} - Sticky bit
  \end{itemize}
\end{itemize}
\begin{itemize}
  \item \textbf{Primjer 8}
  \begin{itemize}
    \item[] \shell{\$ chmod 5754 file1}
    \item[] \shell{\$ ls -l file1}
    \item[] \shell{-rwsr-xr-T 1 pero users 0 Jan  4 23:23 file1}
    \item \vspace{1em} Alternativno:
    \item[] \shell{\$ chmod u=rwxs,g=rx,o=rt file1}
  \end{itemize}
\end{itemize}
\end{frame}

\section{Zadani mode}
\begin{frame}[t]
\frametitle{Zadani mode (1)}
\begin{itemize}
  \item Kreiranjem novog objekta on poprima zadani mode
  \item Definira ga trenutni filesystem i procesi koji kreiraju objekt
\end{itemize}
\begin{itemize}
  \item Primjenom \textbf{umask} mogu se ograničiti dozvole koje postavljaju nadređeni procesi
  \item umask ima isti format kao i mode, no s različitim značenjem bitova
  \begin{itemize}
    \item \textbf{1} - Isključuje dozvolu na poziciji bita
    \item \textbf{0} - Ne mijenja dozvolu koju je postavio nadležni proces
  \end{itemize}
\end{itemize}
\end{frame}

\begin{frame}[t]
  \frametitle{Zadani mode (2)}
\begin{itemize}
  \item Naredbom \shell{umask} se mijenja trenutni umask
  \begin{itemize}
    \item \textbf{Bez argumenata} - ispisuje trenutnu vrijednost u oktalnom obliku
    \item \textbf{Argument -S} - ispisuje trenutnu vrijednost u simboličkom obliku
    \item \textbf{Argument 4 oktalne znamenke} - mijenja vrijednost umaska
    \item[] \hspace{1em} Prva oktalna znamenka je za specijalne modove
  \end{itemize}
\end{itemize}
\begin{itemize}
  \item U datoteci s popisom montiranih datotečnih sustava, \shell{/etc/fstab} se mogu navesti tri vrste maski
  \begin{itemize}
    \item \textbf{umask} - Odnosi se na sve vrste datoteka
    \item \textbf{fmask} - Odnosi se na sve regularne datoteke
    \item \textbf{dmask} - Odnosi se na sve direktorije
  \end{itemize}
  \item Ove vrste maski se mogu navesti i prilikom korištenja naredbe \shell{mount}
\end{itemize}
\end{frame}
  
\section{Promjena vlasnika}
\begin{frame}[t]
\frametitle{Promjena vlasnika}
\begin{itemize}
  \item Promjena vlasnika objekta obavlja se naredbom \shell{chown}
  \begin{itemize}
    \item[] \shell{chown <korisnik> <objekt>}
  \end{itemize}
  \item Promjena grupe objekta obavlja se naredbom \shell{chgrp}
  \begin{itemize}
    \item[] \shell{chgrp <grupa> <objekt>}
    \item[ili] \shell{chown :<grupa> <objekt>}
  \end{itemize}
\end{itemize}
\vfill
\begin{itemize}
  \item Moguće je istovremeno promijeniti korisnika i grupu
  \begin{itemize}
    \item[] \shell{\$ chown <korisnik>:<grupa> <objekt>}
    \item[] \vspace{1em} \shell{\$ chown <korisnik>: <objekt>}
  \item[] Postavlja korisnika i grupu koja odgovara primarnoj grupi korisnika
  \end{itemize}
\end{itemize}
\end{frame}

\end{document}
