\documentclass[table,usenames,dvipsnames] {beamer}

\usepackage[english]{babel}
\usepackage[utf8]{inputenc}
\usepackage{listings}
\usepackage{datetime}
\usepackage{graphics}
\usepackage{fancybox}
\usepackage{color}
\usepackage{courier}
\usepackage[normalem]{ulem}
\usepackage{tikz}
\usepackage{verbatim}
\usepackage{multirow}
\usepackage{fancyvrb}
\usetikzlibrary{shapes,arrows}
\usetheme{CambridgeUS}
\usecolortheme{seagull}
% Changing of bullet foreground color not possible if {itemize item}[ball]
\DefineNamedColor{named}{Purple}{cmyk}{0.52,0.97,0,0.55}
\setbeamertemplate{itemize item}[triangle]
\setbeamercolor{title}{fg=Purple}
\setbeamercolor{frametitle}{fg=Purple}
\setbeamercolor{itemize item}{fg=Purple}
\setbeamercolor{section number projected}{bg=Purple,fg=white}
\setbeamercolor{subsection number projected}{bg=Purple}

\renewcommand{\dateseparator}{.}
\newcommand{\todayiso}{\twodigit\day \dateseparator \twodigit\month \dateseparator \the\year}
\newcommand{\shell}[1]{\texttt{#1}}

\title{Osnove korištenja operacijskog sustava Linux}
\subtitle{05. Korisnici i dozvole}
\author[Lucija Petricioli, Josip Žuljević]{Lucija Petricioli, Josip Žuljević\\{\small Nositelj: doc. dr. sc. Stjepan Groš}}
\institute[FER]{Sveučilište u Zagrebu \\
				Fakultet elektrotehnike i računarstva}
				
\date{\todayiso}

\begin{document}
    %\beamerdefaultoverlayspecification{<+->}
{
\setbeamertemplate{headline}[] % still there but empty
\setbeamertemplate{footline}{}

\begin{frame}
\maketitle
\end{frame}
}

\begin{frame}
\frametitle{Sadržaj}
\tableofcontents
\end{frame}

\section{Terminal i višekorisnički sustav}
\begin{frame}[t]
\frametitle{Osnovni pojmovi (1)}
\begin{itemize}
  \item Linux je višekorisnički operacijski sustav
  \item Uloga višekorisničkog rada u OS-u
  \begin{itemize}
    \item Zaštita privatnosti
    \item Specifične postavke i podaci
    \item Sprečavanje zlouporabe
    \item Pravedna raspodjela resursa
  \end{itemize}
\end{itemize}
\end{frame}

\begin{frame}[t]
\frametitle{Osnovni pojmovi (2)}
\begin{itemize}
	\item Terminal -- U/I naprava za komunikaciju korisnika s računalom
	\item Nekada fizički uređaj, danas programski emulatori
	\item Omogćuju korisniku prikaz \textit{ljuske} - Npr. \shell{bash}
	\begin{itemize} {\small
		\item O Ljuskama detaljnije u idućem predavanju
	} \end{itemize}
  \item Prijava na sustav
  \begin{itemize}
  	\item Prijava lozinkom ili drugim vjerodajnicama
  	\item Odmah po prijavi na sustav korisnik je smješten u svoj matični direktorij
  \end{itemize}
  \item Odjava sa sustava
  \begin{itemize}
  	\item Iz bash ljuske ostvaruje se naredbom \shell{logout}, \shell{exit} ili kombinacijom CTRL+D
  	\item Terminal otvara upit za prijavu novog korisnika
  \end{itemize}
\end{itemize}
\end{frame}

\begin{frame}[t]
\frametitle{Osnovni pojmovi (3)}
  \begin{itemize}
  	\item Terminali su predstavljeni datotekama uređaja:
    \item[] \shell{tty0, tty1, tty2, \ldots}
    \begin{itemize}
    	\item Terminalima upravlja upravljački program - \shell{getty}
    	\item Kod modernih - virtualnih - terminala korisnik može s istog mjesta koristiti više terminala
	  \item Između terminala se prebacuje sa Ctrl+Alt+F1\ldots F7
    \end{itemize}
    \item[] \shell{pts/N}
    \begin{itemize}
      \item Označavaju pseudoterminale - programski emulirane
      \item[] Npr. \shell{gnome-terminal}
    \end{itemize}
  \end{itemize}
\end{frame}

\section{Baza korisnika}
\begin{frame}[t]
	\frametitle{Baza passwd (1)} 
	\begin{itemize}
		\item Temeljna datoteka s korisnicima je \shell{/etc/passwd}
		\begin{itemize} 
			\item Povezuje korisničko ime i UID
			\item Nekada je u njoj bila i lozinka
			\item Vrlo loše sa sigurnosne strane - ne može se zabraniti njeno čitanje jer mnoštvo aplikacija ovisi o podacima u toj datoteci
		\end{itemize}
		\item Sadrži jedan zapis po liniji oblika
		\begin{itemize} {\footnotesize
			\item[] \hspace{-2em} Korisničko ime:Lozinka:UID:GID (primarna grupa):Info:Matični direktorij:Korisnička ljuska
			\item[] \shell{root:x:0:0:root:/root:/bin/bash}
		} \end{itemize}
		\item Uređivanje naredbom \shell{vipw}
	\end{itemize}
\end{frame}

\begin{frame}[t]
	\frametitle{Baza passwd (2)}
	\begin{itemize}
		\item[] Mnogi korisnici navedeni u passwd datoteci ne moraju biti interaktivni korisnici. Neke korisnike koriste servisi koji ne trebaju izravno logiranje u ljusku.
	\end{itemize}
	\begin{itemize}
		\item Matični direktorij korisnika ne mora biti zadan ako se ne radi o interaktivnom korisniku
		\item Ljuska određuje koji se program koristi prilikom prijave korisnika
		\begin{itemize}
	 		\item[] \shell{/bin/bash} - Moguća vrijednost za interaktivnog korisnika
			\item[] \shell{/bin/false} - Moguća vrijednost za korisnika bez mogućnosti prijave na sustav
		\end{itemize}
	\end{itemize}
\end{frame}

\begin{frame}[t]
	\frametitle{Baza shadow}
	\begin{itemize}
		\item Ako u passwd bazi na mjestu lozinke stoji \shell{x} tada se sigurnosni podaci o korisniku nalaze u datoteci \shell{/etc/shadow}
		\begin{itemize}
			\item Sadrži kriptirane lozinke, te dodatne podatke o njihovom trajanju
			\item Čitljiva je isključivo root korisniku
		\end{itemize}
	\end{itemize}
	\begin{itemize}
		\item Sadrži jedan zapis po liniji oblika
		\begin{itemize} {\footnotesize
			\item[] \hspace*{-3em} Korisničko ime:Lozinka:Polja s dodatnim podacima
			\item[] \shell{root:T3RqrzxU1MAH3F3wtuQu/:13284:0:99999:7:::}
		} \end{itemize}
	\end{itemize}
\end{frame}

\begin{frame}[t]
\frametitle{Naredba \shell{who}}
\begin{itemize}
  \item Naredba može prikazati podatke o korisniku
  \item Primjer ispisa
  \begin{itemize}
    \item[] \shell{\$ who}
    \begin{table}[h]
    \begin{flushleft}
    \shell{
    \begin{tabular}{l l l l l}
      cetko & tty7 & 2010-11-11 & 12:01 & (:0)  \\
      cetko & pts/0 & 2010-11-11 & 17:08 & (:0) \\
      cetko & pts/1 & 2010-11-11 & 17:08 & (:0) \\ 
      cetko & pts/2 & 2010-11-11 & 17:12 & (:0)  
    \end{tabular} }
    \end{flushleft}
    \end{table}
  \end{itemize}
  \item Poseban oblik naredbe \shell{who} je \shell{who am i}
  \begin{itemize}
    \item Ispisuje tko je trenutni korisnik na trenutnom terminalu
  \end{itemize}
  \item Varijanta te naredbe je \shell{whoami}
  \begin{itemize}
    \item Ispisuje samo korisničko ime
  \end{itemize}
\end{itemize}
\end{frame}

\begin{frame}[t]
\frametitle{Naredba \shell{finger}}
\begin{itemize}
  \item Drugi način prikaza trenutno aktivnih korisnika
  \item Prikazuje trenutno logirane korisnike, ili prikazuje 
          detaljnije podatke o nekom korisniku
\end{itemize}
\begin{itemize}
   \item Prikazuje dodatne podatke
   \begin{itemize}
   	\item Iz \shell{Info} polja u passwd bazi
   	\item Čita ih iz datoteka \shell{.project} i \shell{.plan} u matičnom direktoriju
   \end{itemize}
   \item Ako joj zadamo parametar pretražuje korisnika
   \begin{itemize}
   	\item Pretraživanje se obavlja po korisničkom imenu i pravom imenu
   \end{itemize}
\end{itemize}
\end{frame}

\begin{frame}[t]
\frametitle{Naredba \shell{w}}
\begin{itemize}
  \item Primjer ispisa
    \begin{table}[h]\footnotesize
    \begin{flushleft}
    \shell{
    \begin{tabular}{l l l l l l l l}
      USER & TTY & FROM & LOGIN@ & IDLE & JCPU & PCPU & WHAT  \\
      cetko & tty7 & :0 & 12:01 & 5:32m & 3:45 & 9.67s & awesome \\
      cetko & pts/0 & :0 & 17:29 & 3:21 & 0.33s & 0.33s & bash \\
      cetko & pts/1 & :0 & 7:31 & 1:06 & 0.33s & 0.33s & bash \\
      cetko & pts/5 & :0 & 17:23 & 0.00s & 0.32s & 0.00s & w
    \end{tabular} }
    \end{flushleft}
    \end{table}
\end{itemize}
\end{frame}

\begin{frame}[t]
\frametitle{root} 
\begin{itemize}
  \item Operacijski sustav korisnike identificira preko jedinstvenog identifikatora
  \begin{itemize}
	  \item[] \textbf{UID (User ID)}
  \end{itemize}
\end{itemize}
\begin{itemize}
  \item Jedan korisnik se posebno tretira 
  \begin{itemize}
    \item[] \bf{root} \hspace{2em} UID=0
  \end{itemize}
\end{itemize}
\begin{itemize}
  \item root može sve
  \begin{itemize}
    \item Nije preporučljivo ulogiravati se i/ili raditi kao root!!!
    \item Raditi kao običan korisnik pa tek kad je nužno prebaciti se na 
          root korisnika
  \end{itemize}
\end{itemize}
\end{frame}

\begin{frame}[fragile]
\frametitle{sudo}
\begin{itemize}
  \item Za privremeno dobivanje administrativnih ovlasti koristi se naredba (prefiks) \shell{sudo}
  \item sudo mogu izvršiti svi korisnici prema dozvolama definiranima u datoteci
  \item[] {\small\shell{/etc/sudoers}}
  \item Uređivanje naredbom \shell{visudo}
\end{itemize}
\vfill
\footnotesize
\begin{Verbatim}
dino     ALL = (ALL) ALL
dominik  marvin,magrathea = (ALL) root
josip    marvin = /usr/bin/passwd [[:alpha:][:digit]]*,
                  !/usr/bin/passwd dino
%kset    ALL = NOPASSWD: /sbin/umount /media/cdrom0
\end{Verbatim}
\end{frame}

\begin{frame}[t]
\frametitle{Mijenjanje korisnika} 
\begin{itemize}
  \item Vrlo bitna naredba \shell{su} (engl. \emph{switch user})
  \item Dva bitna oblika naredbe
  \begin{itemize}
    \item \shell{su <korisnicko ime>}
    \item \shell{su - <korisnicko ime>}
  \end{itemize}
  \item Bez argumenata mijenja korisnika u root
\end{itemize}
\end{frame}

\section{Grupe}
\begin{frame}[t]
\frametitle{Grupe (1)}
\begin{itemize}
  \item Korisnici se grupiraju u korisničke grupe
  \begin{itemize}
  	\item Administracija korisnika
  	\item Dijeljenje podataka
  	\item Zajedničke dozvole
  \end{itemize}
  \item Svaki korisnik ima
  \begin{itemize}
    \item[] \textbf{Primarnu grupu}
    \begin{itemize}
      \item Zapisana u datoteci \shell{etc/passwd} 
    \end{itemize}
    \item[] \textbf{Sekundarne grupe}
    \begin{itemize}
      \item  Sve grupe kojima korisnik pripada
    \end{itemize}
  \end{itemize}
\end{itemize}
\end{frame}

\begin{frame}[t]
	\frametitle{Grupe (2)}
	\begin{itemize}
		\item Slično kao i za korisnike za grupe se koristi groups baza u datoteci \shell{/etc/group}
		\item Sadrži jedan zapis po liniji oblika
		\begin{itemize} {\footnotesize
			\item[] \hspace{-2em} Ime grupe:Lozinka:GID:Popis korisnika
			\item[] \shell{cdrom:x:24:linux,dominik,dino}
		} \end{itemize}
	  \item Grupe također imaju posebnu datoteku za lozinke \shell{/etc/gshadow}
	\end{itemize}
	\begin{itemize}
	  \item Operacijski sustav i s grupama radi preko jedinstvenog identifikatora
	  \begin{itemize}
		\item[] \textbf{GID (Group ID)}
	  \end{itemize}
	  \item Naredbom \shell{id} saznajemo sve grupe u koje korisnik pripada
	  \begin{itemize}
		\item[] \shell{uid=1000(user) gid=1000(user) groups=1000(user),4(adm)...}
	  \end{itemize}
      \item Privremena prijava u druge grupe naredbom \shell{newgrp}
	 \end{itemize}
\end{frame}


\section{Upravljanje korisnicima}
\begin{frame}[t]
\frametitle{Upravljanje korisnicima}
\begin{itemize}
  \item Osnovne operacije s korisnicima
  \begin{itemize}
    \item Dodavanje novog korisnika
    \begin{itemize}
      \item[] \shell{adduser}
    \end{itemize}
    \item Promjena lozinke korisnika
    \begin{itemize}
      \item[] \shell{passwd}
    \end{itemize}
    \item Promjena podataka o korisniku
    \begin{itemize}
      \item[] \shell{usermod}
    \end{itemize}
    \item Uklanjanje korisnika
    \begin{itemize}
      \item[] \shell{deluser}
    \end{itemize}
  \end{itemize}
\end{itemize}
\begin{itemize}
	\item Analogne naredbe postoje i za grupe
	\begin{itemize}
		\item[] \shell{groupadd, groupmod, groupdel}
	\end{itemize}
\end{itemize}
\end{frame}

\begin{frame}[t]
\frametitle{Upravljanje korisnicima}
\begin{itemize}
  \item Stvaranje novog korisnika
  \begin{itemize}
    \item[] \shell{\$ adduser <korisnik>}
  \end{itemize}
  \item Dodavanje korisnika postojećoj grupi
  \begin{itemize}
  	\item[] \shell{\$ usermod -aG <grupa> <korisnik>}
    \item[ili] \shell{\$ adduser <korisnik> <grupa>}
  \end{itemize}
  \item Stvaranje nove grupe
  \begin{itemize}
    \item[] \shell{\$ addgroup <grupa>}
    \item[ili] \shell{\$ adduser --group <grupa>}
  \end{itemize}
\end{itemize}
\end{frame}

\begin{frame}[t]
\frametitle{Promjena podataka o korisniku}
\begin{itemize}
  \item Promjena podataka o korisniku
  \begin{itemize}
    \item Mogu se mjenjati svi podaci
    \begin{itemize}
      \item[] \shell{usermod <opcije> <username>}
    \end{itemize}
    \item Promjena ljuske, opcija \shell{-s <shell>}
    \item Promjena matičnog direktorija, opcija \shell{-d <dir>}
  \end{itemize}
\end{itemize}
\begin{itemize}
  \item Ljuska korisnika može se promijeniti i  naredbom \shell{chsh}
  \item Naredba \shell{chfn} mijenja dodatne podatke o korisnicima
  \begin{itemize}
  	\item[] Finger podaci - \shell{Info} polje
  \end{itemize}
  \item Lozinka se mijenja naredbom \shell{passwd}
\end{itemize}
\end{frame}

\begin{frame}[t]
\frametitle{Upravljanje korisnicima}
\begin{itemize}
  \item Brisanje kreiranog korisnika
  \begin{itemize}
    \item[] \shell{\$ deluser <korisnik>}
  \end{itemize}
  \item Brisanje korisnika iz grupe
  \begin{itemize}
    \item[] \shell{\$ deluser <korisnik> <grupa>}
  \end{itemize}
  \item Brisanje grupe
  \begin{itemize}
  	\item[] \shell{\$ delgroup <grupa>}
    \item[ili] \shell{\$ deluser --group <grupa>}
  \end{itemize}
\end{itemize}
\end{frame}

\begin{frame}[t]
\frametitle{Upravljanje korisnicima}
\begin{itemize}
  \item Kod stvaranja korisnika se može definirati lokacija matičnog 
        direktorija i njegovo brisanje zajedno sa korisnikom
  \item Navedene naredbe su sučelja drugih naredbi
  \begin{itemize}
    \item[] \shell{adduser} $\Rightarrow$ \shell{useradd}
    \item[] \shell{deluser} $\Rightarrow$ \shell{userdel}
    \item[] \shell{addgroup} $\Rightarrow$ \shell{groupadd}
    \item[] \shell{delgroup} $\Rightarrow$ \shell{groupdel}
  \end{itemize}
  \item Sve prethodne akcije se mogu napraviti i navedenim naredbama
\end{itemize}
\end{frame}

\begin{frame}[t]
\frametitle{Upravljanje korisnicima}
\begin{itemize}
  \item Ako kod stvaranja korisnika nisu definirani parametri, koriste se 
        postavke u \shell{/etc/adduser.conf}
  \item U matičnom direktoriju se stvaraju predefinirane datoteke
  \begin{itemize}
     \item Raspored početnih datoteka je definiran u direktoriju \shell{/etc/skel} (engl. \emph{skeleton})
  \end{itemize}
\end{itemize}
\begin{itemize}
  \item Zadatak
  \begin{itemize}
    \item Proučiti opcije u datoteci \shell{/etc/adduser.conf}
    \item Izlistati direktorij \shell{/etc/skel} i matični direktorij
  \end{itemize}
\end{itemize}
\end{frame}

\begin{frame}[t]
	\frametitle{Naredbe}
	\begin{table}[h]
		%  \rowcolors{2}{White}{LightGray}
		\begin{tabular}{|c|l|}
			\hline
			\rowcolor{BlueViolet!20}Naredba & Opis \\
			\hline
			Ctrl+D & odjava iz terminala \\
			\hline
			logout & odjava iz terminala \\
			\hline
			who & prikazuje podatke o korisniku \\
			\hline
			who am i & ispisuje korisnika u trenutnom terminalu \\
			\hline
			whoami & ispisuje isključivo korisničko ime korisnika u terminalu \\
			\hline
			finger & ispisuje trenutno aktivne korisnike \\
			\hline
			su & izmjena korisnika \\
			\hline
			newgrp & prijava u drugu grupu \\
			\hline
			usermod & izmjena podataka o korisniku \\
			\hline
			passwd & promjena korisničke lozinke \\
			\hline
		\end{tabular}
	\end{table}
\end{frame}

\end{document}