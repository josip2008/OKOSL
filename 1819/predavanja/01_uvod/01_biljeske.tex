
\begin{document}
\section{Operacijski sustav}
Što je uopće operacijski sustav? Za nas(korisnike) operacijski sustav je
resource manager. Njegova zadaća je na pametan način dodjeljivati resurse.
Što znači na pametan način? Znači da siguran i učinkovit način mora osigurati
da će korisnik dobiti ono što i očekuje (ni više ni manje od toga).

Primjerice, od operacijskog sustava očekujemo da nam osigura da samo jedna
aplikacija smije narediti printeru ispis. Pored toga, očekujemo i u neku ruku
lakše korištenje kompjutera. Poprilično bi teško bilo navigirati file system
ako umjesto datoteka vidimo memorijske lokacije na disku (što podaci u suštini
i jesu).

Kernel je taj koji i cijeloj priči brine o pametnom aspektu, on osigurava
višekorisnički i višezadaćni način rada, on brine o virtualnoj memoriji i
slično. Sve to bi bilo poprilično beskorisno ako na neki način kernel ne bi
gledao prema user-spaceu. To gledanje kernela prema user-spaceu se obično
naziva Linux API kojeg predstavljaju sistemski pozivi (read, open, write, ..).


\end{document}
