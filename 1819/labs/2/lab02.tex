\documentclass[12pt,a4paper]{article}
\usepackage[croatian]{babel}
\usepackage[utf8]{inputenc}
\DeclareUnicodeCharacter{B0}{\degree}
\DeclareUnicodeCharacter{2192}{\rightarrow}
\DeclareUnicodeCharacter{2193}{\downarrow}

\usepackage[top=20mm]{geometry}
\usepackage{enumitem}

\usepackage{listings}
\usepackage{amsfonts}
\usepackage{amssymb}
\usepackage{gensymb}
\usepackage{hyperref}

\usepackage{enumitem}

\newcommand{\shell}[1]{\texttt{\textbf{#1}}}
\renewcommand*{\familydefault}{\sfdefault}
\renewcommand*{\sfdefault}{lmss}
\begin{document}
	\title{Laboratorijska vježba 2\\{\small Osnove korištenja operacijskog sustava Linux}\vspace{-2em}}
	\maketitle
	Za svaki zadatak potrebno je napisati jednu bash skriptu. Kako biste lakše demonstrirali rješenja preporučamo da nakon svakog ključnog koraka u zadatku prikažete rezultate izvršavanja skripte. U tome vam mogu pomoći sljedeće naredbe:
	\begin{description}[leftmargin=!,labelwidth=4em,itemsep=0em]
		\item[\shell{clear}] Čisti sadržaj terminala
		\item[\shell{read -p}] Čita podatak s tipkovnice; Zaustavlja izvođenje skripte
		\item[\shell{less}] Pager
	\end{description}

	\subsection*{Zadatak 1}
	Sljedeće je točke potrebno riješiti nad datotekom \shell{/usr/include/stdio.h} koristeći regularne izraze (engl. \textit{regular expressions}).
	\begin{itemize}
		\item Prebrojite redove koji su uvučeni.
		\item Prebrojite koliko u datoteci postoji komentara i blokova s komentarima.
	\end{itemize}

	\subsection*{Zadatak 2}
	\begin{itemize}
		\item Ispišite sve datoteke čiji sadržaj počinje slovom \shell{c} iza kojeg slijede 2 slova, točka i barem jedno slovo iza točke \\ \textit{Primjer: \shell{cba.d} }
		\item Ispišite sve datoteke koje ne sadrže mala slova od \shell{a} do \shell{k} (abecedno), a imaju barem jednu znamenku.
	\end{itemize}

	\subsection*{Zadatak 3}
	Korištenjem \textit{here document} sintakse kreirajte datoteku \shell{Top10} sadržaja:
	\begin{verbatim}
	Linux Mint 17.2
	Ubuntu 15.10
	Debian GNU/Linux 8.2
	Mageia 5
	Fedora 23
	openSUSE Leap 42.1
	Arch Linux
	CentOS 7.2-1511
	PCLinuxOS 2014.12
	Slackware Linux 14.1
	FreeBSD
	\end{verbatim}
	Nad datotekom je potrebno izvršiti sljedeće zadatke, koristeći naredbu \shell{sed}:
	\begin{itemize}
		\item Obrisati sve redove koji sadrže imena distribucija bez brojeva.
		\item Brojeve verzija prebaciti na početak reda.
		\item Sva slova promijeniti u mala.
		\item Sve samoglasnike prebaciti u velika slova.
		\item Sortirati datoteku po numeričkoj vrijednosti brojeva na početku retka.
	\end{itemize}
	Ispišite sadržaj datoteke prije i poslije obrade.

	\subsection*{Zadatak 4}
	\begin{itemize}
		\item Pronađite sve Python datoteke na sustavu (hint: ekstenzija \shell{.py}) i ispišite nazive svih funkcija iz njih (hint: ključna riječ \shell{def}).
		\item Pronađite sve C datoteke u sustavu. Ispišite sve pretprocesorske naredbe iz njih.
		\item Pronađite sve datoteke u sustavu koje sadrže niz \shell{include}. Ispišite redni broj retka u kojemu se nalazi pronađeni niz.
	\end{itemize}

	\subsection*{Zadatak 5}
	\begin{itemize}
    \item Datotekama koje imaju naziv oblika \shell{PNG-DDMMYYYY} (dan, mjesec, godina) promijenite naziv u \shell{DD\_MM\_YYYY.png} \\ \textit{Primjer: \shell{PNG-07092015} $\longrightarrow$ \shell{07\_09\_2015.png}}
	\end{itemize}

    \subsection*{Zadatak 6}
    Ispiti su gotovi, a to se osjeti i na vašem Downloads direktoriju. U njemu
    se nalazi obilje datoteka skinutih uoči međuispita. Budući da ste sad već
    vični bash eksperti odlučili ste počistiti svoj Downloads direktorij sa
    stilom, s isključenim mišem, koristeći bash skriptu koju ste napisali u
    opskurnom text editoru!

    Da rezimiramo, zadatak je ``pospremiti'' Downloads folder tako da datoteke iz
    njega smjestimo tamo gdje im je i mjesto. Kako znamo gdje je mjesto kojim
    datotekama? Sve datoteke koje imaju underscore '\_' imaju isti obrazac
    imenovanja, a to je:

    \begin{itemize}
        \item[] \shell{IPRED\_Ime\_datoteke.eks}
    \end{itemize}

    Drugim riječima, prvo polje imena označavat će ime predmeta kojem datoteka
    pripada, dok će ostala polja biti ime datoteke na čijem kraju se nalazi
    ekstenzija koja će određivati tip.

    Za početak smo vam pripremili umjetni Downloads folder na kojem ćete
    testirati svoja rješenja. Da bi stvar bila još bolja, možete ga izbrisati
    i ponovno napraviti baš onakvim kakav je bio u trenutku $t = -0$. Skriptu
    s rješenjima nazovite \texttt{sort.sh}.

    Pretpostavka je da u trenutku pokretanja skripte za postavljanje zadatka,
    vaš trenutni direktorij izgleda ovako:

    \begin{verbatim}
$ tree .
.
|-- generate_files.sh
|-- lab02.pdf
|-- sort.sh
|-- setup.sh
    \end{verbatim}

    Zadatak testirajte u \shell{/tmp/OKOSL} direktoriju zbog lakšeg eksperimentiranja.
    Također, zbog lakšeg eksperimentiranja pripremili smo skripte koje će vam pomoći
    postaviti zadatak tako da ne trošite vrijeme na brisanje netočnih pokušaja.
    Za početak je potrebno generirati naš umjetni \shell{downloads} direktorij pokretanjem
    skripte (\shell{setup.sh}). Skriptu \shell{setup.sh} i skriptu za generiranje direktorija
    (\shell{generate\_files.sh}) dohvatite koristeći naredbu \shell{wget} ili pak obični browser s poveznice: \url{https://marvin.kset.org/~okosl/lab2/}.

    \subsubsection*{Pseudokod u nastavku potrebno je implementirati u okviru jedne
    bash skripte}
    \begin{verbatim}
smjesti se u downloads folder;
za svaki file u direktoriju:
    ako file ima _ u imenu:
        ime_predmeta = prvo polje imena;
        ako ne postoji direktorij ../ime_predmeta:
            napravi ga;
        premjesti file u direktorij ../ime_predmeta;
    inače
        ako ne postoji direktorij ../razonoda:
            napravi ga;
        premjesti file u direktorij ../razonoda;
    \end{verbatim}

    Skripta radi super! Direktorij \shell{downloads} je napokon čist. Jedini problem
    je da je direktorij \shell{razonoda} sad pun različitog sadržaja... U
    direktoriju se nalaze knjige (format \shell{.pdf} i \shell{.epub}), pjesme
    (format \shell{.mp3}) i  slike (format \shell{.jpg} i \shell{.jpeg}).
    Budući da i dalje ne znate gdje vam je miš odlučili ste čišćenje odraditi
    do kraja sa stilom, pa ćete proširiti skriptu iz prvog dijela tako da sortira
    i direktorij \shell{razonoda}!

    \subsubsection*{Pseudokod u nastavku potrebno je implementirati ili u okviru
    skripte ranije ili u okviru nove skripte}
\begin{verbatim}
smjesti se u razonoda folder;
za svaki file u folderu:
    ako je ekstenzija .pdf ili .epub:
        ako ne postoji folder ../knjige:
            napravi ga
        premjesti file u folder ../knjige
    ako je ekstenzija .jpg ili .jpeg:
        ako ne postoji folder ../slike:
            napravi ga
        premjesti file u folder ../slike
    ako je ekstenzija .mp3:
        ako ne postoji folder ../muzika:
            napravi ga
        premjesti file u folder ../muzika
\end{verbatim}

    Napomena: Možete biti sigurni da datoteke u direktoriju \shell{razonoda} sve imaju samo
    jednu točku i to neposredno prije ekstenzije.

\end{document}
