\documentclass{exam}
\usepackage[croatian]{babel}
\usepackage[utf8]{inputenc}
\usepackage{xcolor}
\usepackage{enumerate}
\usepackage{listings}
\usepackage{amsmath}
\usepackage{hyperref}
\newcommand{\shell}[1]{\texttt{#1}}
\lstset{basicstyle=\ttfamily,
  showstringspaces=false,
  commentstyle=\color{red},
  keywordstyle=\color{blue}
}

\author{Osnove korištenja operacijskog sustava Linux}
\date{\today}
\pagenumbering{arabic}

\begin{document}
\title{Domaća zadaća - 02}
\maketitle

\paragraph{Kratke upute}
\subparagraph{}
Rješenje domaće zadaće potrebno je upisati u bash skriptu imena \shell{dz02-rjesenja.sh}. Prije rješenja svakog (pod)zadatka potrebno je dodati komentar s rednim brojem (pod)zadatka. Ako zadatak ima podzadatke, potrebno je dodati oznaku početka samo podzadataka, ne i cijelog zadatka. Primjerice, ako drugi zadatak nema podzadataka a treći ima dva podzadatka, isječak skripte bi trebao izgledati otprilike ovako:
\begin{lstlisting}[language=bash]
# Zadatak 2
<neke naredbe>

# Zadatak 3.a
<neke naredbe>

# Zadatak 3.b
<neke naredbe>
\end{lstlisting}

Svaki (pod)zadatak \textbf{mora} imati svoju komentar-oznaku u skripti, neovisno o tome je li zadatak riješen ili ne.



\subsection*{Zadatak 1.}
  \begin{enumerate}[(a)]
    \item Stvorite datoteku koja u prvom redu sadrži vaš JMBAG koristeći naredbu
      \texttt{echo} i preusmjeravanje. Datoteku nazovite
      \texttt{personal\_info.dat}, a potom ju ispišite.
    \item Pomoću naredbe \texttt{cat} i here dokumenta dodajte vaše ime i
      prezime u redak ispod JMBAG-a. Tako konstruirani dokument potom ispišite
      na ekran.
    \item Pomoću naredbe \shell{tee} dodajte "okosl" u redak ispod imena i
      prezimena i ispišite rezultat.
\end{enumerate}

  \subsection*{Zadatak 2.}
  \begin{enumerate}[(a)]
    \item[] Za sve točke ovog zadatka iskoristite datoteku \texttt{korisnici.dat}. Imena u datoteci nisu case-sensitive. Svaki redak datoteke ima format
    \item[] \texttt{<broj retka>:<ime>:<id korisnika>}
    \item[] Radni direktorij zadatka je \shell{/tmp/OKOSL}
    \item Sortirajte datoteku po broju retka i rezultat zapišite u \texttt{korisnici.sortirano.dat}. Ispišite rezultat.
    \item Ispišite sva jedinstvena imena (koja se ne ponavljaju) u datoteku\\\texttt{jedistveni\_korisnici.dat}.  Ispišite rezultat.
    \item Imena koja se ponavljaju više puta zapišite u datoteku\\\texttt{nejedinstveni\_korisnici.dat}.  Ispišite rezultat.
  \end{enumerate}
  \subsection*{Zadatak 3.}
  \begin{enumerate}[(a)]
  	\item Prebrojite koliko riječi iz datoteke \texttt{/usr/share/dict/words} sadrže tekst \shell{ping}.
  	\item U datoteku \texttt{yous.dat} izdvojite sve riječi iz datoteke \texttt{/usr/share/dict/words} koje sadrže tekst \shell{you}, a zatim ju ispišite.
  	\item Prebrojite koliko je takvih riječi.
  	\item Udvostručite sadržaj datoteke \texttt{yous.dat}, i ispišite datoteku s udvostručenim sadržajem.
  \end{enumerate}
  \subsection*{Zadatak 4.}
  Locirajte datoteku \texttt{jedinstveni\_korisnici.dat} stvorenu u zadatku 2 koristeći, redom, naredbe
  \begin{enumerate}[(a)]
    \item \texttt{find}
    \item \texttt{locate}
\end{enumerate}

    \subsection*{Zadatak 5.}
    \begin{enumerate}[(a)]
        \item Koristeći operator $\rvert$ odnosno \textit{pipe} na stdout ispišite sve datoteke iz svog home direktorija kojima je zadnje vrijeme promjene (eng. \textit{modification time}) bilo u 10. mjesecu. (\textit{HINT: grep na "Oct"\footnote{Ili,
    jasno, \shell{Lis} od "Listopad" ili nešto treće, ovisno o lokalizacijskim
    postavkama.}
}).
        \item Output dobiven u prošloj naredbi sortirajte s obzirom na sedmi
          stupac u ispisu prema numeričkoj vrijednosti brojeva i ispišite ga.
        \item Finalni output naredbe upišite u datoteku
          \shell{/tmp/sortiran\_home.txt}. Ispišite rezultat.
    \end{enumerate}
    
\underline{\textbf{Za one koji žele znati više:}}
\begin{itemize}
    \item Dokažite da \shell{/tmp/sortiran\_home.txt} postoji koristeći uvjetno izvođenje s naredbom \shell{test -f}.

\end{itemize}
\end{document}
